\subsection{Niet-functionele eisen}
\label{nietfunctioneleeisen}

\subsubsection{Gebruikersinterface en menselijke factoren}
Gebruikers zijn onderzoekers en artsen. 
De gebruikers hebben weinig tijd om zich een nieuw systeem aan te leren. 
Mede daardoor moet de interface makkelijk te begrijpen en eenvoudig zijn. 
Wel is het mogelijk om vak specifieke termen te gebruiken doordat alle gebruikers hiermee bekend zijn.

\subsubsection{Documentatie}
Er hoeft alleen een installatieprocedure geschreven te worden. Deze moet beschrijven hoe het systeem ge\"{i}nstalleerd moet worden en hoe de communicatie met de hoofdserver ingesteld wordt. Door de eenvoud van de uiteindelijke applicatie hoeft er geen handleiding voor de gebruikers te worden geschreven.

\subsubsection{Hardwarezaken}
Er is voor het draaien van de applicatie ��n centrale server nodig, vanwege de eisen aan beveiliging moet deze uiteindelijk slechts te bereiken zijn via het interne netwerk van het EMC. De server moet de mogelijkheid hebben om de gebruikte services te draaien. 

\subsubsection{Prestatie-eigenschappen}
Doordat de gebruikers over het algemeen te weinig tijd hebben om met het systeem bezig te zijn moet het systeem snel reageren en mag een handeling geen minuten duren.

\subsubsection{Foutafhandeling en extreme omstandigheden}
Het systeem is slechts een hulpmiddel in een project. 
Er hangen geen kritieke systemen en/of beslissingen af van het functioneren van het systeem. 

\subsubsection{Systeemaanpassingen}
Het systeem wordt zodanig geschreven dat het makkelijk uitbreidbaar is en uiteindelijk grotendeels hergebruikt kan worden in latere fases van het project. Het systeem wordt daarnaast open source aangeleverd zodat ook anderen kunnen helpen aan het systeem.

\subsubsection{Fysieke omgeving}
Vanwege de gevoeligheid van de data die via de software op de server wordt opgeslagen, moet de fysieke toegang tot de server tot een minimum beperkt worden. Aangezien het EMC meer met gevoelige informatie werkt, moet dit geen probleem zijn.

\subsubsection{Beveiliging}
Zoals hierboven al genoemd wordt er gewerkt met gevoelige informatie en moet het systeem hier op een juiste manier mee omgaan. Er zal authenticatie vereist zijn voordat met de software gewerkt kan worden.
