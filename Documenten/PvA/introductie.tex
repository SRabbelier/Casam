\section{Introductie}

Dit document vormt de basis voor alle planning en de inhoud van ons project.
Het is het eerste product dat we afleveren en hierin wordt het verloop van het project en eisen voor het product vastgelegd.

\subsection{Aanleiding}
%Hierbij wordt ingegaan op de oorzaak die geleid heeft tot het formuleren van de projectopdracht, het effectueren ervan en de omstandigheden waaronder dit Plan van Aanpak tot stand is gekomen.
%Indien van belang zal worden verwezen naar gevoerde gesprekken en referenties.

Het project waar we aan zullen werken is het C.A.S.A.M.-project, dat ons via Charl Botha van de TUDelft is toevertrouwd.
C.A.S.A.M. staat voor Computer Assistend Surgical Anatomy Mapping.
Via contacten met het EMC kwam hij op het spoor van Anton Kerver die een groep Informaticastudenten zocht om hem te helpen bij een (software)project waarover hij wil publiceren.
Tijdens een uitgebreid gesprek op 1 april hebben we de bedoeling en mogelijkheden van ons project doorgenomen.
De hieruit voortvloeiend projectomschrijving is in dit document opgenomen.

\subsection{Accordering en bijstelling}
%Hier wordt opgenomen op welke wijze het Plan van Aanpak wordt goedgekeurd en bijgesteld.
%De voortgang en bijstellingen op het plan worden vastgesteld middels de voortgangsrapportage.
%Nadat voorgestelde wijzigingen zijn goedgekeurd is impliciet het Plan van Aanpak bijgesteld.
%Het actuele Plan van Aanpak wordt op deze wijze gevormd door het oorspronkelijke Plan van Aanpak en de voortgangsrapportages.

Als we zelf denken dat dit Plan van Aanpak volledig is en klopt met de wens van de opdrachtgever zullen we met hem dit document doorspreken, zodat we zeker weten dat de wederzijdse verwachtingen overeenstemmen en omschreven liggen in dit plan.
Dit gesprek zal plaatsvinden op 15 april.
Dit document zelf zal al op 10 april afgeleverd worden aan de opdrachtgever.

Elke week zal er een voortgangsgesprek zijn waarin we het product dat er dan is tonen aan de opdrachtgever.
De wijzigingen die deze gesprekken kunnen betekenen voor het Plan van Aanpak zullen live worden bijgewerkt.
Dit betekent dat dit document met de projectvoortgang zal meegroeien zodat achteraf het Plan van Aanpak ook alle wijzigingen zal bevatten en dus het meest accuraat is.


\subsection{Toelichting op de opbouw van het plan}
%Hierin wordt de structuur van het plan toegelicht.

Dit document is gestructureerd volgens het gegeven voorbeeld vanaf de TU.
Eerst omschrijven we de opdracht inhoudelijk in hoofdstuk \ref{projectopdracht}, waarna we in hoofdstuk \ref{aanpak} onze aanpak omschrijven.
In hoofdstuk \ref{projectinrichting} staat hoe wij ons project in zullen richten, waarna in in hoofdstuk \ref{plannen} de planning gegeven wordt.
In hoofdstuk \ref{kwaliteit} geven we aan hoe we over de kwaliteit van het product zullen waken.
We sluiten af met overige plannen in hoofdstuk \ref{overige} die niet met tijd en middelen te maken hebben. Aan het eind zijn bijlagen opgenomen. 
