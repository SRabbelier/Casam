\section{Aanpak}
\label{aanpak}
\subsection{Risico analyse}

Het product maakt gebruik van meerdere software pakketten,
hier komt van nature een bepaald risico bij kijken in verband met het integreren in het product.
Op grond van een risicoanalyse kunnen de volgende maatregelen worden genomen:
\begin{description}
    \item[preventie] het voorkomen dat iets gebeurt of het verminderen van de kans dat het gebeurt;
    \item[repressie] het beperken van de schade wanneer een bedreiging optreedt;
    \item[acceptatie] geen maatregelen, men accepteert de kans en het mogelijke gevolg van een bedreiging;
    \item[manipulatie] het wijzigen van parameters in de berekening om tot een gewenst resultaat te komen.
\end{description}

De bedoeling van een risicoanalyse is dat er na de analyse wordt vastgesteld op welke wijze de risico's beheerst kunnen worden, of teruggebracht tot een aanvaardbaar niveau. \cite{wiki:risico_analyse}

We hebben de volgende risico factoren geidentificeerd:
\begin{description}
    \item[python] niet iedereen is bekend met python en heeft dus tijd nodig om op snelheid te komen
    \item[django] django is een template taal en vereist zeer weinig nieuwe kennis
    \item[database] het gebruik van een beveiligde database kan problemen veroorzaken
    \item[python image libraries] er zijn vanuit de Delft Computer Graphics group, niemand heeft deze eerder gebruikt
    \item[nose test] testen met een nieuw framework is altijd even wennen, en kan veel tijd kosten
\end{description}

Aan de hand van deze risico's hebben we de volgende milestone's opgesteld:
\begin{enumerate}
    \item Een eenvoudige applicatie die mbv Django een 'Hello world' pagina kan laten zien
    \item CRUD (Create, Read, Update, Delete) van data naar een database mbv Django
    \item Verwerken van images mbv python image libraries
    \item Tests voor iedere milestone
\end{enumerate}


%In het hoofdstuk Aanpak wordt de brug geslagen tussen het afgebakende resultaat en de inrichting van het project,
door middel van beantwoording van de ``hoe''-vraag.
Doel is om door middel van Aanpak overeenstemming te verkrijgen over de te volgen weg, om te komen tot het gewenste resultaat.

Per eindresultaat wordt aangegeven welke activiteiten zullen worden uitgevoerd en eventueel welke tussenresultaten worden opgeleverd.
Tevens wordt hierbij ingegaan op het waarom van de gekozen oplossing.
Daarbij wordt verwezen naar de cruciale succesfactoren, de resultaten van de uitgevoerde risico analyse,
en de geformuleerde eisen en beperkingen ten aanzien van proces, resultaat en kwaliteit.
Als de projectmanager daarin op basis van de uitgangspositie, cruciale succesfactoren,
risico analyse of kwaliteitseisen onduidelijkheid of onvolledigheid vaststelt, geeft hij aan hoe hij met deze zaken omgaat.

De projectmanager zal het project structureren en faseren,
om aan te geven in welke globale stappen hij de projectopdracht denkt uit te voeren.

Bij het structureren groepeert hij de gewenste eindresultaten primair naar algemene aandachtsgebieden.
De volgende algemene aandachtsgebieden worden onderkend:

\begin{itemize}
  \item Ontwikkeling resultaat
  \item Voorbereiding gebruik, dit zijn de activiteiten die samenhangen met het (her)inrichten van de gebruikersorganisatie
  \item Voorbereiding beheer, dit zijn de activiteiten die samenhangen met het (her)inrichten van de beheerorganisatie
  \item Acceptatie gebruik, het voorbereiden en uitvoeren van de gebruikers-acceptatie
  \item Acceptatie beheer, het voorbereiden en uitvoeren van de beheeracceptatie
  \item Kennis, dit zijn de activiteiten die samenhangen met het opbouwen van materiekennis met betrekking tot het resultaat
  (ook van het gebruik en het beheer ervan) en de activiteiten die samenhangen met de overdracht van deze kennis,
  naar de staande organisatie.
\end{itemize}

Afhankelijk voor het type project worden de voor het project te hanteren aandachtsgebieden afgeleid uit de algemene aandachtsgebieden.
Ook spelen andere criteria bij het structureren een rol, bijvoorbeeld:

\begin{itemize}
  \item risico factoren
  \item cruciale succesfactoren
  \item kwaliteitseisen
\end{itemize}

Naast het structureren zal het project tevens in de tijd worden gefaseerd om formele meet- en beslismomenten te verkrijgen.
De fasering wordt gericht op de beslissingen die de opdrachtgever wil nemen
en vindt ondermeer plaats op basis van invoeringstijdstip of product.

Per aandachtsgebied en verdere onderverdeling, wordt aangegeven door welke activiteiten het eindresultaat wordt bereikt,
wat de samenhang van de activiteiten is en welke tussenresultaten worden opgeleverd binnen c.q. buiten de projectopdracht.
Indien nodig kan de samenhang gevisualiseerd worden in de vorm van een eenvoudig netwerkplan zonder kwantitatieve gegevens.

Conform de structuur en fasering wordt dit hoofdstuk in paragrafen opgedeeld.

