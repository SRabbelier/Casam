\section{Plannen}
In het hoofdstuk plannen wordt de resultante vastgelegd van het evenwicht tussen activiteiten, tijd,
geld en middelen teneinde de opdracht te kunnen uitvoeren.
De verschillende paragrafen worden als volgt ingevuld:

\subsection{Normen en aannames}
Hierbij worden de gehanteerde normen, aannames en veronderstellingen zowel ten aanzien van de schattingen,
als ten aanzien van planning vermeld, zoveel mogelijk per eenheid verbijzonderd.
Deze kunnen afkomstig zijn uit geraadpleegde literatuur aangevuld met ``ervaringscijfers''.

\subsection{Activiteitenplan}
In deze paragraaf worden de uit te voeren activiteiten beschreven.
De detaillering hiervan is sterk afhankelijk van de opdrachtformulering en de fase waarin het project zich bevindt.
Per activiteit wordt weergegeven de benodigde inspanning, de tijdsduur,
de samenhang met andere activiteiten en het benodigde resourceniveau.

\subsection{Mijlpalen-/Productenplan}
Het mijlpalenplan geeft de meet- of beslismomenten weer.
Hierbij worden de meest belangrijke momenten voor toetsing en sturing benadrukt.
Het productenplan geeft de momenten weer waarop de (tussen)producten zullen worden opgeleverd en geaccepteerd.

\subsection{Resourceplan}
Het resourceplan verschaft duidelijkheid over personele en overige middelen.
Het plan geeft weer over welke perioden inzet benodigd is. Bij de personele middelen wordt tevens het niveau van de resource aangegeven.

\subsection{Financieel plan}
In deze paragraaf wordt inzicht gegeven in de kosten (mensen, middelen en overig) van het project.
Aangegeven worden de resources die in de planning zijn opgenomen,
de hiervoor gehanteerde tarieven en de hieruit resulterende verwachte kosten.\
