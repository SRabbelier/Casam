\section{Introductie}

%OMSCHRIJVING SECTIE
%De introductie geeft een inleiding op de inhoud van dit document. Van tevoren is het goed om te weten wat de aanleiding van de totstandkoming tot en met de huidige status van het project is.  

In dit hoofdstuk maken we een korte introductie op \casam en wat we gaan doen. 
 
\subsection{Aanleiding}

%OMSCHRIJVING SECTIE
%Hierbij wordt ingegaan op de oorzaak die geleid heeft tot het formuleren van de projectopdracht, het effectueren ervan en de omstandigheden waaronder dit Plan van Aanpak tot stand is gekomen. Indien van belang zal worden verwezen naar gevoerde gesprekken en referenties.

Het project is onderdeel van het \casamproject (Computer Assisted Surgical Anatomy Mapping). 
Bij \casam is het de bedoeling dat anatomische vraagstellingen van verschillende klinische afdelingen sneller, makkelijker en vooral duidelijker kunnen worden onderzocht.
Vanuit \casam kwamen wij in contact met Anton Kerver die een groep Informaticastudenten zocht om hem te helpen bij een (software)project waarover hij wil publiceren. 
Tijdens een uitgebreid gesprek op 1 april hebben we de bedoeling en de mogelijkheden van ons project doorgenomen (zie sectie \ref{notulen}).
Op basis hiervan zijn wij begonnen met het schrijven van dit document.
\subsection{Accordering en bijstelling}

%OMSCHRIJVING SECTIE
%Bovendien wordt aangegeven hoe de goedkeuring en de bijstelling van dit Plan van Aanpak geregeld is.

Als het eerste concept van het Plan van Aanpak af is wordt dit naar de opdrachtgever gestuurd.
Na overleg over dit document zal het Plan van Aanpak bijgesteld worden en zal er een definitieve versie ontstaan, zo weten we zeker dat de wederzijdse verwachtingen overeenstemmen en zijn omschreven. 

Elke week zal er een voortgangsgesprek zijn met de opdrachtgever, de wijzigingen die deze gesprekken betekenen voor het Plan van Aanpak worden hierin bijgewerkt.

\subsection{Toelichting op de opbouw van het plan}

%OMSCHRIJVING SECTIE
%Als laatste wordt uiteengezet hoe de rest van dit document gestructureerd is.

Dit document is gestructureerd aan de hand de gegeven voorbeelden van de Technische Universiteit Delft met aanvullingen naar ons eigen inzicht.
In sectie \ref{projectopdracht} omschrijven we de opdracht.
In sectie \ref{aanpak_en_tijdsplanning} beschrijven wij onze aanpak en tijdsplanning.
Daarna wordt in de sectie \ref{projectinrichting} uitgelicht hoe wij ons project in zullen delen.
Ten slotte zal in sectie \ref{kwaliteit} aandacht worden besteed aan de kwaliteitswaarborging van het product.
\pagebreak