\section{Introductie}

%OMSCHRIJVING SECTIE
%De introductie geeft een inleiding op de inhoud van dit document. Van tevoren is het goed om te weten wat de aanleiding van de totstandkoming tot en met de huidige status van het project is. 

Dit document vormt de basis voor alle planning en de inhoud van ons project.
Het is het eerste product dat we afleveren en hierin worden het verloop van het project en de eisen voor het product vastgelegd.

\subsection{Aanleiding}

%OMSCHRIJVING SECTIE
%Hierbij wordt ingegaan op de oorzaak die geleid heeft tot het formuleren van de projectopdracht, het effectueren ervan en de omstandigheden waaronder dit Plan van Aanpak tot stand is gekomen. Indien van belang zal worden verwezen naar gevoerde gesprekken en referenties.

Het project waar we aan zullen werken is onderdeel van het \casamproject (Computer Assisted Surgical Anatomy Mapping). 
Bij \casam is het de bedoeling dat anatomische vraagstellingen van verschillende klinische afdelingen sneller, makkelijker en vooral duidelijker kunnen worden onderzocht.
Vanuit \casam kwamen wij in contact met Anton Kerver, die een groep Informaticastudenten zocht om hem te helpen bij een (software)project waarover hij wil publiceren.
Tijdens een uitgebreid gesprek op 1 april \ref{notulen} hebben we de bedoeling en mogelijkheden van ons project doorgenomen.
De hieruit voortvloeiende projectomschrijving is in dit document opgenomen.

\subsection{Accordering en bijstelling}

%OMSCHRIJVING SECTIE
%Bovendien wordt aangegeven hoe de goedkeuring en de bijstelling van dit Plan van Aanpak geregeld is.

Als het eerste concept van het Plan van Aanpak af is, wordt dit naar de opdrachtgever gestuurd.
Daarna zal er overleg zijn over dit document, zodat we zeker weten dat de wederzijdse verwachtingen overeenstemmen en omschreven zijn in dit Plan van Aanpak.
Na dit gesprek zal het Plan van Aanpak bijgesteld worden en zal er een definitieve versie ontstaan.

Elke week zal er een voortgangsgesprek zijn met de opdrachtgever.
De wijzigingen die deze gesprekken betekenen voor het Plan van Aanpak worden hierin bijgewerkt.

\subsection{Toelichting op de opbouw van het plan}

%OMSCHRIJVING SECTIE
%Als laatste wordt uiteengezet hoe de rest van dit document gestructureerd is.

Dit document is gestructureerd aan de hand van het gegeven voorbeeld van de TU Delft met aanvullingen naar ons eigen inzicht.
Eerst omschrijven we de opdracht in sectie \ref{projectopdracht}.
In sectie \ref{aanpak_en_tijdsplanning} beschrijven wij onze aanpak en tijdsplanning.
Daarna wordt in de sectie \ref{projectinrichting} uitgelicht hoe wij ons project in zullen delen.
Ten slotte zullen we in sectie \ref{kwaliteit} aandacht besteden aan de kwaliteitsborging van het product.
