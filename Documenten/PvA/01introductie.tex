\section{Introductie}

Dit document vormt de basis voor alle planning en de inhoud van ons project.
Het is het eerste product dat we afleveren en hierin worden het verloop van het project en eisen voor het product vastgelegd.

\subsection{Aanleiding}
%Hierbij wordt ingegaan op de oorzaak die geleid heeft tot het formuleren van de projectopdracht, het effectueren ervan en de omstandigheden waaronder dit Plan van Aanpak tot stand is gekomen.
%Indien van belang zal worden verwezen naar gevoerde gesprekken en referenties.

Het project waar we aan zullen werken is onderdeel van het \casamproject (Computer Assistend Surgical Anatomy Mapping). 
Bij \casam is het de bedoeling dat anatomische vraagstellingen van verschillende klinische afdelingen sneller, makkelijker en vooral duidelijker worden behandeld.
Vanuit \casam kwamen wij in contact met Anton Kerver, die een groep Informaticastudenten zocht om hem te helpen bij een (software)project waarover hij wil publiceren.
Tijdens een uitgebreid gesprek op 1 april hebben we de bedoeling en mogelijkheden van ons project doorgenomen.
De hieruit voortvloeiende projectomschrijving is in dit document opgenomen.

\subsection{Accordering en bijstelling}
%Hier wordt opgenomen op welke wijze het Plan van Aanpak wordt goedgekeurd en bijgesteld.
%De voortgang en bijstellingen op het plan worden vastgesteld middels de voortgangsrapportage.
%Nadat voorgestelde wijzigingen zijn goedgekeurd is impliciet het Plan van Aanpak bijgesteld.
%Het actuele Plan van Aanpak wordt op deze wijze gevormd door het oorspronkelijke Plan van Aanpak en de voortgangsrapportages.

Als het eerste concept van het Plan van Aanpak af is, wordt dit naar de opdrachtgever gestuurd.
Daarna zal er overleg zijn over dit document zodat we zeker weten dat de wederzijdse verwachtingen overeenstemmen en omschreven zijn in dit plan.
Na dit gesprek zal het Plan van Aanpak bijgesteld worden en zal er een definitieve versie ontstaan.

Elke week zal er een voortgangsgesprek zijn met de opdrachtgever.
De wijzigingen die deze gesprekken betekenen voor het Plan van Aanpak worden hierin bijgewerkt.

\subsection{Toelichting op de opbouw van het plan}
%Hierin wordt de structuur van het plan toegelicht.

Dit document is gestructureerd aan de hand van het gegeven voorbeeld van de TU Delft.
Eerst omschrijven we de opdracht in sectie \ref{projectopdracht}.
In sectie \ref{aanpak} beschrijven wij onze aanpak.
Daarna worden in de secties \ref{projectinrichting} en \ref{plannen} uitgelicht hoe wij ons project in zullen delen.
Ten slotte zullen we in sectie \ref{kwaliteit} aandacht besteden aan de kwaliteitsborging van het product.
