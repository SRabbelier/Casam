\section{Projectinrichting en voorwaarden}
\label{projectinrichting}
%OMSCHRIJVING SECTIE
%Het doel van projectinrichting is het zichtbaar maken van de wijze waarop de projectmanager van plan is het project in te richten om de opdracht uit te voeren volgens de voorgestelde aanpak. De volgende punten zullen daar meestal onderdeel van zijn. Ook andere afspraken over de projectinrichting worden hier opgenomen.

In dit hoofdstuk zal een globale beschrijving worden gegeven van de projectinrichting. 

Het project zal onderverdeeld moeten worden in eenduidige onderdelen, zodat er niet alleen een planning gemaakt kan worden,
maar deze planning ook aangepast kan worden als blijkt dat er meer of minder tijd nodig is dan gedacht.
Door het project in onderdelen te verdelen kunnen er onderdelen aan de planning worden toegevoegd,
of juist weggelaten, zonder dat de kwaliteit van het product hierdoor vermindert.


\subsection{Organisatie}
%OMSCHRIJVING SECTIE
%Hoe zijn de verantwoordelijkheden onder de projectleden verdeeld.

Tijdens het project zullen we ons opdelen in groepen van twee die steeds samen aan een feature werken.
Deze groepen worden zo vaak mogelijk afgewisseld zodat ieder teamlid ervaring heeft met alle onderdelen van het product. Dit gebeurt volgens de Agile methode \cite{wiki:agile}.
We zullen voor verschillende taken een hoofdverantwoordelijke aanstellen, zodat voor die onderdelen het overzicht goed bewaard wordt.
Een aantal taken die een hoofdverwantwoordelijke nodig zouden kunnen hebben zijn:
\begin{itemize}
    \item Database ontwerp en integriteit
    \item User Interface Design en Usability
    \item Import en export van data
\end{itemize}


\subsection{Personeel}
%OMSCHRIJVING SECTIE
%Welke eisen worden er aan het personeel gesteld qua tijdsinvestering en vaardigheden.

Er wordt van ieder teamlid verwacht dat hij of zij gemiddeld 40 uur per week bezig is met het project, gedurende 10 weken.

De studenten van de TU Delft die aan dit project werken, zijn:
\begin{itemize}
	\item Sjors van Berkel
	\item Bastiaan Bijl
	\item Jaap den Hollander
	\item Sverre Rabbelier
	\item Ben Sedee
	\item Noeska Smit
\end{itemize}
Al deze personen zitten in hun derde studiejaar van de opleiding Technische Informatica, en zitten dus op hetzelfde niveau. 
De overige personen die hun medewerking verlenen aan dit project zijn:
\begin{description}
	\item[Dr. Botha] Begeleider van de TU Delft
	\item[Anton Kerver] Student aan het EMC, nauw betrokken bij het \casamproject
	\item[Dr. Kleinrensink] projectleider van het \casamproject
\end{description}

Het gehele project begint op 6 april 2009, en eindigt op 19 juni 2009.

\subsection{Administratieve procedures}
%OMSCHRIJVING SECTIE
%Welke administratieve procedures zullen gebruikt worden om het project te monitoren en in goede banen te leiden.

Binnen en rond het project zijn de volgende procedures van toepassing:

\begin{itemize}
    \item Documenten worden geschreven in Latex.
    \item Wekelijkse genotuleerde vergaderingen op de TU.
    \item Wekelijkse feedback van de opdrachtgever.
    \item Via ticketsysteem worden de taken verdeeld
\end{itemize}

\subsection{Financing}
%OMSCHRIJVING SECTIE
%Op welke wijze zal de projectmanager zorgen voor de financiering van de benodigde resources en wat zijn de totale kosten (het gaat er hier dus niet om wat de opdrachtgever moet betalen, maar hoe de financiering intern geregeld is).

Financiering is niet van toepassing op dit project, omdat alle te gebruiken apparatuur en software al tot onze beschikking staat.

Aangezien dit project een bachelorproject is van studenten van de TU Delft, en alle resources die benodigd zijn worden geleverd door de TU Delft en/of het \casamproject, zijn er met dit project geen kosten gemoeid. Er staat waarschijnlijk ook geen vergoeding tegenover.

\subsection{Rapportering}
%OMSCHRIJVING SECTIE
%Hier wordt uitgelegd op welke manier gecommuniceerd zal worden naar de opdrachtgever.

We starten elke dag met een korte vergadering waarin we de voortgang en openstaande taken bespreken. Aan het begin van elke week vindt er een grotere vergadering plaats die genotuleerd wordt en waarbij belangrijke beslissingen kunnen worden gemaakt. Bij elke grote fase zal de bijbehorende documentatie beschikbaar worden gemaakt aan de opdrachtgever en de begeleider.

\subsection{6.	Technische resources}
%OMSCHRIJVING SECTIE
%Als laatste wordt omschreven hoe gebruikt wordt gemaakt van technische resources en hoe de werkplekken eruit zullen zien. Alle benodigde middelen worden besproken.

De niet personele middelen waar wij gebruik van maken zijn:
\begin{itemize}
	\item SVN Server van de TU Delft
	\item Trac Server van de TU Delft
\end{itemize}
Alle voor dit project gebruikte software, is vrij gegeven onder een open source licentie.
