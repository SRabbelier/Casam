\section{Projectinrichting en voorwaarden}
\label{projectinrichting}

In dit hoofdstuk zal een globale beschrijving worden gegeven van de projectinrichting. Aan de hand van het OPAFIT-model zullen we de verschillende onderdelen bespreken. 

\subsection{Projectinrichting}

Het project zal onderverdeeld moeten worden in eenduidige onderdelen zodat er een planning gemaakt kan worden,
maar deze planning ook aangepast kan worden als blijkt dat er meer of minder tijd nodig is dan gedacht.
Door het project in onderdelen te verdelen kuinnen er onderdelen aan de planning worden toegevoegd,
of juist weggelaten, zonder dat de kwaliteit van het product hierdoor verminderd.

\subsection{OPAFIT}

In deze sectie zullen we volgens de OPAFIT aspecten de projectrichting bespreken.

\subsubsection{Organisatie}

Tijdens het project zullen we ons opdelen in groepen van twee die steeds samen aan een feature werken.
Deze groepen worden zo vaak mogelijk afgewisseld zodat ieder teamlid ervaring heeft met alle onderdelen van het product.
We zullen voor verschillende taken een hoofdverantwoordelijke aanstellen zodat voor die onderdelen het overzicht goed bewaard wordt.
Een aantal taken die een hoofdverwantwoordelijke nodig zouden kunnen hebben zijn:
\begin{itemize}
    \item Database ontwerp en integriteit
    \item User Interface Design en Usability
    \item Import/export van data
\end{itemize}

\subsubsection{Personeel}

Er wordt verwacht van ieder teamlid gemiddeld 40 uur per week bezig is met het project, gedurende 10 weken.

\subsubsection{Administratieve procedures}

Binnen en rond het project zijn de volgende procedures van toepassing:

\begin{itemize}
    \item Documenten worden geschreven in Latex.
    \item We programeren volgens de Agile methode \cite{wiki:agile}.
    \item Wekenlijkse feedback van de opdrachtgever
\end{itemize}

\subsubsection{Financing}

Financiering is niet van toepassing op dit project, omdat alle te gebruiken apparatuur al tot de beschikking staat.

\subsubsection{Informatie}




% Afhankelijk van de opdracht en de organisatie komen de OPAFIT aspecten aan de orde:
% \begin{description}
%  \item[Organisatie] waarbij aangegeven wordt hoe de projectorganisatie eruit komt te zien inclusief taken en verantwoordelijkheden.
%  Deze worden per persoon en per rol gesteld
%  \item[Personeel] waarbij de eisen aan de gewenste inzet en beschikbaarheid van personeel worden aangegeven,
%  zoals condities voor het betrekken van personeel, per groep de vereiste vakkennis, skills gerelateerd aan de plannen
%  \item[Administratieve procedures] waarin alle binnen en rond het project van toepassing zijnde procedures worden genoemd
%  \item[Financing] alle financi\"ele zaken worden hier behandeld, bij voorkeur met verwijzingen of, bij afwezigheid,
%  expliciet opgenomen zoals tariefwijzigingen, facturering, subcontractors, btw en dergelijke;
%  \item[Informatie] waarbij ingegaan wordt op alle informatie rond het project, overleg- en rapportagestructuren;
%  \item[Techniek] waarbij wordt ingegaan op de voorgestelde inrichting qua hard- en software, werkplekken, hulpmiddelen en dergelijke.
%\end{description}

\subsection{Voorwaarden aan opdrachtnemer}

%Opsomming van voorwaarden, die gerealiseerd dienen te worden door de opdrachtnemer om het project volgens plan te kunnen uitvoeren.
%Deze voorwaarden zijn gerelateerd aan en aanvullend op de inrichtingsaspecten.

\subsection{Voorwaarden aan opdrachtgever}

%idem als 4.2, echter met opdrachtgever i.p.v. opdrachtnemer.

\subsection{Voorwaarden aan derden}

%idem als 4.2, echter met derden i.p.v. opdrachtnemer.

