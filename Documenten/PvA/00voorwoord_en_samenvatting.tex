\addcontentsline{toc}{section}{Voorwoord}
\begin{center}
\textbf{\large Voorwoord}
\end{center}

%OMSCHRIJVING SECTIE
%Uitleg wat dit document is en welke rol het vervult. Het Plan van Aanpak is een soort van contract waarin wordt vastgelegd wat het 'probleem' is dat opgelost gaat worden in het project. Er wordt vastgelegd waaraan de opdrachtgever en de projectleden (hierin geleidt door de projectmanager) moeten voldoen. Bovendien wordt beschreven hoe het project gefaseerd zal worden en worden overige afspraken vastgelegd.


In dit document worden de plannen beschreven voor het Bachelor\-project \casam van de Tech\-nische Universiteit Delft.
Dit project vindt plaats in samenwerking met de onderzoeksafdeling Neuro\-wetenschap\-pen-Anatomie van het Erasmus Medisch Centrum.

%\setcounter{section}{-1}
\addcontentsline{toc}{section}{Management Samenvatting}
\section*{Samenvatting}

%OMSCHRIJVING SECTIE
%In de samenvatting worden de belangrijkste conclusies uit het rapport samengevat, staan kort de voorwaarden die moeten gelden voor de geldigheid van dit Plan van Aanpak en worden alle momenten waarop input van de opdrachtgever vereist is opgesomd.

De opdracht is om een systeem te ontwerpen in opdracht van de \casam onderzoeksgroep.
Dit gaan we doen onder de voorwaarde dat het in 10 weken gebouwd wordt, en dat het eindproduct op een gebruiksvriendelijke manier de mogelijkheid geeft om onderzoeken van preparaten op te slaan en hier op bewerkingen uit te voeren.
