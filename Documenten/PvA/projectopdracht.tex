\section{Projectopdracht}

In dit hoofdstuk wordt de gewenste verandering in beeld gebracht.
De opdracht wordt afgebakend, door middel van het beantwoorden van de ``waarom'', de ``waarover'' en de ``wat''-vragen.

Deze zaken worden in ``opdrachtgevers bewoordingen'' aan de orde gebracht.
De paragrafen worden als volgt ingevuld:

\subsection{Projectomgeving}
Wat is het beschouwingsgebied?
Hierin wordt een schets gegeven van het beschouwingsgebied in termen van organisatie eenheden en bedrijfsprocessen.
Tevens wordt aangegeven wat de problemen en oorzaken zijn die aanleiding geven tot de ontwikkeling van het resultaat.


\subsection{Doelstelling project}
Waarom heeft de opdrachtgever het resultaat nodig en wat wil de opdrachtgever met het resultaat bereiken?
In deze paragraaf wordt een beschrijving gegeven van de doelstellingen van het te ontwikkelen resultaat,
zoals aangegeven door de opdrachtgever.
Met name wordt hierbij de koppeling gelegd naar bedrijfsprocessen.
Hierbij is het van belang om te weten, waarop de opdrachtgever wordt afgerekend.
Iedere doelstelling wordt zo mogelijk onderbouwd door kwalitatieve en kwantitatieve gegevens.

\subsection{Opdrachtformulering}
Wat is de projectopdracht?
Waarover gaat het project procesmatig (afbakening)?
Deze paragraaf beschrijft de opdracht, voortvloeiend uit de doelstelling, zoals aangegeven door de opdrachtgever.
Hierbij wordt expliciet aangegeven welke zaken wel en welke zaken niet tot de verantwoordelijkheid van het project worden gerekend.
Aangegeven wordt ook of het een resultaat- of een inspanningsverplichting betreft.

\subsection{Op te leveren producten en diensten}
Wat is het resultaat van het project?
Waarover gaat het project inhoudelijk (afbakening)?
Deze paragraaf bevat de specificatie van de op te resultaten zoals aangegeven door de opdrachtgever.
Dit is een nadere uitwerking van de projectopdracht, zoals aangegeven bij de opdrachtformulering.

\subsection{Eisen en beperkingen}
In deze paragraaf worden de acceptatiecriteria en beperkingen vermeld,
die de opdrachtgever stelt aan het resultaat en de eisen en beperkingen die gesteld worden aan de gebruikte resources en aan de wijze,
waarop het resultaat tot stand komt.
De eisen moeten zo nauwkeurig mogelijk worden gekwantificeerd.
Indien mogelijk worden er ook prioriteiten vastgesteld.

\subsection{Cruciale succesfactoren}
Deze paragraaf beschrijft de door de opdrachtgever onderkende en specifiek voor deze opdracht geldende cruciale succesfactoren.
Het moet zowel de opdrachtgever als de projectmanager duidelijk zijn welke maatregelen mogelijk zijn,
c.q. door beiden genomen moeten worden om deze factoren te be\"invloeden.


Van groot belang is de juiste interpretatie van een aantal onderdelen van de Projectopdracht:

De Doelstelling geeft aan wat het achterliggende doel is van het starten van het project.
Dit kan het doorvoeren van een organisatorische verandering zijn op uiteenlopende niveau's,
zoals klant-, bedrijfs-, effici\"entie-, of middelenniveau.

De Opdrachtformulering geeft weer door welk middel de opdrachtgever de gewenste doelstelling denkt te bereiken.

De Eisen en beperkingen geven aan welke eisen de opdrachtgever stelt aan het eindresultaat en het procesmatige verloop van de opdracht.

De Cruciale Succesfactoren geven aan,  welke door de opdrachtnemer be\"invloedbare zaken er, vanuit de opdrachtgever gezien, essentieel zijn om het resultaat zo goed mogelijk te laten aansluiten bij de te bereiken doelstelling.

