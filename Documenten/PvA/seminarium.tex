\documentclass{article}

\usepackage[dutch]{babel}
\usepackage{a4wide}
\usepackage{graphicx}

\title{CASAM}

\author{Groep Bachelorproject 42}
\date{}

\pagestyle{empty}

\begin{document}

\maketitle
\thispagestyle{empty}

\begin{abstract}

Op een beknopt aantal pagina's worden de ``high-light's'' uit het Plan van Aanpak weergegeven,
aangevuld met de geldigheidscondities van het Plan van aanpak.
Tot slot wordt een overzicht van alle beslispunten voor de opdrachtgever gegeven.

\end{abstract}

\section{Introductie}

De introductie is gericht op het Plan van Aanpak en het tot stand komen ervan.
Ingegaan wordt op de volgende aspecten:

\subsection{Aanleiding}
Hierbij wordt ingegaan op de oorzaak die geleid heeft tot het formuleren van de projectopdracht,
het effectueren ervan en de omstandigheden waaronder dit Plan van Aanpak tot stand is gekomen.
Indien van belang zal worden verwezen naar gevoerde gesprekken en referenties.

\subsection{Accordering en bijstelling}
Hier wordt opgenomen op welke wijze het Plan van Aanpak wordt goedgekeurd en bijgesteld.
De voortgang en bijstellingen op het plan worden vastgesteld middels de voortgangsrapportage.
Nadat voorgestelde wijzigingen zijn goedgekeurd is impliciet het Plan van Aanpak bijgesteld.
Het actuele Plan van Aanpak wordt op deze wijze gevormd door het oorspronkelijke Plan van Aanpak en de voortgangsrapportages.

\subsection{Toelichting op de opbouw van het plan}
Hierin wordt de structuur van het plan toegelicht.

\section{Projectopdracht}

In dit hoofdstuk wordt de gewenste verandering in beeld gebracht.
De opdracht wordt afgebakend, door middel van het beantwoorden van de ``waarom'', de ``waarover'' en de ``wat''-vragen.

Deze zaken worden in ``opdrachtgevers bewoordingen'' aan de orde gebracht.
De paragrafen worden als volgt ingevuld:

\subsection{Projectomgeving}
Wat is het beschouwingsgebied?
Hierin wordt een schets gegeven van het beschouwingsgebied in termen van organisatie eenheden en bedrijfsprocessen.
Tevens wordt aangegeven wat de problemen en oorzaken zijn die aanleiding geven tot de ontwikkeling van het resultaat.


\subsection{Doelstelling project}
Waarom heeft de opdrachtgever het resultaat nodig en wat wil de opdrachtgever met het resultaat bereiken?
In deze paragraaf wordt een beschrijving gegeven van de doelstellingen van het te ontwikkelen resultaat,
zoals aangegeven door de opdrachtgever.
Met name wordt hierbij de koppeling gelegd naar bedrijfsprocessen.
Hierbij is het van belang om te weten, waarop de opdrachtgever wordt afgerekend.
Iedere doelstelling wordt zo mogelijk onderbouwd door kwalitatieve en kwantitatieve gegevens.

\subsection{Opdrachtformulering}
Wat is de projectopdracht?
Waarover gaat het project procesmatig (afbakening)?
Deze paragraaf beschrijft de opdracht, voortvloeiend uit de doelstelling, zoals aangegeven door de opdrachtgever.
Hierbij wordt expliciet aangegeven welke zaken wel en welke zaken niet tot de verantwoordelijkheid van het project worden gerekend.
Aangegeven wordt ook of het een resultaat- of een inspanningsverplichting betreft.

\subsection{Op te leveren producten en diensten}
Wat is het resultaat van het project?
Waarover gaat het project inhoudelijk (afbakening)?
Deze paragraaf bevat de specificatie van de op te resultaten zoals aangegeven door de opdrachtgever.
Dit is een nadere uitwerking van de projectopdracht, zoals aangegeven bij de opdrachtformulering.

\subsection{Eisen en beperkingen}
In deze paragraaf worden de acceptatiecriteria en beperkingen vermeld,
die de opdrachtgever stelt aan het resultaat en de eisen en beperkingen die gesteld worden aan de gebruikte resources en aan de wijze,
waarop het resultaat tot stand komt.
De eisen moeten zo nauwkeurig mogelijk worden gekwantificeerd.
Indien mogelijk worden er ook prioriteiten vastgesteld.

\subsection{Cruciale succesfactoren}
Deze paragraaf beschrijft de door de opdrachtgever onderkende en specifiek voor deze opdracht geldende cruciale succesfactoren.
Het moet zowel de opdrachtgever als de projectmanager duidelijk zijn welke maatregelen mogelijk zijn,
c.q. door beiden genomen moeten worden om deze factoren te be\"invloeden.


Van groot belang is de juiste interpretatie van een aantal onderdelen van de Projectopdracht:

De Doelstelling geeft aan wat het achterliggende doel is van het starten van het project.
Dit kan het doorvoeren van een organisatorische verandering zijn op uiteenlopende niveau's,
zoals klant-, bedrijfs-, effici\"entie-, of middelenniveau.

De Opdrachtformulering geeft weer door welk middel de opdrachtgever de gewenste doelstelling denkt te bereiken.

De Eisen en beperkingen geven aan welke eisen de opdrachtgever stelt aan het eindresultaat en het procesmatige verloop van de opdracht.

De Cruciale Succesfactoren geven aan,  welke door de opdrachtnemer be\"invloedbare zaken er, vanuit de opdrachtgever gezien, essentieel zijn om het resultaat zo goed mogelijk te laten aansluiten bij de te bereiken doelstelling.


\section{Aanpak}

In het hoofdstuk Aanpak wordt de brug geslagen tussen het afgebakende resultaat en de inrichting van het project,
door middel van beantwoording van de ``hoe''-vraag.
Doel is om door middel van Aanpak overeenstemming te verkrijgen over de te volgen weg, om te komen tot het gewenste resultaat.

Per eindresultaat wordt aangegeven welke activiteiten zullen worden uitgevoerd en eventueel welke tussenresultaten worden opgeleverd.
Tevens wordt hierbij ingegaan op het waarom van de gekozen oplossing.
Daarbij wordt verwezen naar de cruciale succesfactoren, de resultaten van de uitgevoerde risico analyse,
en de geformuleerde eisen en beperkingen ten aanzien van proces, resultaat en kwaliteit.
Als de projectmanager daarin op basis van de uitgangspositie, cruciale succesfactoren,
risico analyse of kwaliteitseisen onduidelijkheid of onvolledigheid vaststelt, geeft hij aan hoe hij met deze zaken omgaat.

De projectmanager zal het project structureren en faseren,
om aan te geven in welke globale stappen hij de projectopdracht denkt uit te voeren.

Bij het structureren groepeert hij de gewenste eindresultaten primair naar algemene aandachtsgebieden.
De volgende algemene aandachtsgebieden worden onderkend:

\begin{itemize}
  \item Ontwikkeling resultaat
  \item Voorbereiding gebruik, dit zijn de activiteiten die samenhangen met het (her)inrichten van de gebruikersorganisatie
  \item Voorbereiding beheer, dit zijn de activiteiten die samenhangen met het (her)inrichten van de beheerorganisatie
  \item Acceptatie gebruik, het voorbereiden en uitvoeren van de gebruikers-acceptatie
  \item Acceptatie beheer, het voorbereiden en uitvoeren van de beheeracceptatie
  \item Kennis, dit zijn de activiteiten die samenhangen met het opbouwen van materiekennis met betrekking tot het resultaat
  (ook van het gebruik en het beheer ervan) en de activiteiten die samenhangen met de overdracht van deze kennis,
  naar de staande organisatie.
\end{itemize}

Afhankelijk voor het type project worden de voor het project te hanteren aandachtsgebieden afgeleid uit de algemene aandachtsgebieden.
Ook spelen andere criteria bij het structureren een rol, bijvoorbeeld:

\begin{itemize}
  \item risico factoren
  \item cruciale succesfactoren
  \item kwaliteitseisen
\end{itemize}

Naast het structureren zal het project tevens in de tijd worden gefaseerd om formele meet- en beslismomenten te verkrijgen.
De fasering wordt gericht op de beslissingen die de opdrachtgever wil nemen
en vindt ondermeer plaats op basis van invoeringstijdstip of product.

Per aandachtsgebied en verdere onderverdeling, wordt aangegeven door welke activiteiten het eindresultaat wordt bereikt,
wat de samenhang van de activiteiten is en welke tussenresultaten worden opgeleverd binnen c.q. buiten de projectopdracht.
Indien nodig kan de samenhang gevisualiseerd worden in de vorm van een eenvoudig netwerkplan zonder kwantitatieve gegevens.

Conform de structuur en fasering wordt dit hoofdstuk in paragrafen opgedeeld.

\section{Projectinrichting en voorwaarden}

Geef aan het begin van dit hoofdstuk een korte inleiding, want dat is tof. %Sjors: ehmm??

\subsection{Projectinrichting}

Het doel van projectinrichting is het zichtbaar maken van de wijze waarop de projectmanager,
van plan is het project in te richten om de opdracht uit te voeren volgens de voorgestelde aanpak.
Hierbij zal de gekozen inrichting afhankelijk zijn van de resultaten van de risico analyse,
kwaliteitseisen en de cruciale succesfactoren.

Afhankelijk van de opdracht en de organisatie komen de OPAFIT aspecten aan de orde:
\begin{description}
  \item[Organisatie] waarbij aangegeven wordt hoe de projectorganisatie eruit komt te zien inclusief taken en verantwoordelijkheden.
  Deze worden per persoon en per rol gesteld
  \item[Personeel] waarbij de eisen aan de gewenste inzet en beschikbaarheid van personeel worden aangegeven,
  zoals condities voor het betrekken van personeel, per groep de vereiste vakkennis, skills gerelateerd aan de plannen
  \item[Administratieve procedures] waarin alle binnen en rond het project van toepassing zijnde procedures worden genoemd
  \item[Financing] alle financi\"ele zaken worden hier behandeld, bij voorkeur met verwijzingen of, bij afwezigheid,
  expliciet opgenomen zoals tariefwijzigingen, facturering, subcontractors, btw en dergelijke;
  \item[Informatie] waarbij ingegaan wordt op alle informatie rond het project, overleg- en rapportagestructuren;
  \item[Techniek] waarbij wordt ingegaan op de voorgestelde inrichting qua hard- en software, werkplekken, hulpmiddelen en dergelijke.
\end{description}

\subsection{Voorwaarden aan opdrachtnemer}
Opsomming van voorwaarden, die gerealiseerd dienen te worden door de opdrachtnemer om het project volgens plan te kunnen uitvoeren.
Deze voorwaarden zijn gerelateerd aan en aanvullend op de inrichtingsaspecten.

\subsection{Voorwaarden aan opdrachtgever}

idem als 4.2, echter met opdrachtgever i.p.v. opdrachtnemer.

\subsection{Voorwaarden aan derden}

idem als 4.2, echter met derden i.p.v. opdrachtnemer.


\section{Plannen}
In het hoofdstuk plannen wordt de resultante vastgelegd van het evenwicht tussen activiteiten, tijd,
geld en middelen teneinde de opdracht te kunnen uitvoeren.
De verschillende paragrafen worden als volgt ingevuld:

\subsection{Normen en aannames}
Hierbij worden de gehanteerde normen, aannames en veronderstellingen zowel ten aanzien van de schattingen,
als ten aanzien van planning vermeld, zoveel mogelijk per eenheid verbijzonderd.
Deze kunnen afkomstig zijn uit geraadpleegde literatuur aangevuld met ``ervaringscijfers''.

\subsection{Activiteitenplan}
In deze paragraaf worden de uit te voeren activiteiten beschreven.
De detaillering hiervan is sterk afhankelijk van de opdrachtformulering en de fase waarin het project zich bevindt.
Per activiteit wordt weergegeven de benodigde inspanning, de tijdsduur,
de samenhang met andere activiteiten en het benodigde resourceniveau.

\subsection{Mijlpalen-/Productenplan}
Het mijlpalenplan geeft de meet- of beslismomenten weer.
Hierbij worden de meest belangrijke momenten voor toetsing en sturing benadrukt.
Het productenplan geeft de momenten weer waarop de (tussen)producten zullen worden opgeleverd en geaccepteerd.

\subsection{Resourceplan}
Het resourceplan verschaft duidelijkheid over personele en overige middelen.
Het plan geeft weer over welke perioden inzet benodigd is. Bij de personele middelen wordt tevens het niveau van de resource aangegeven.

\subsection{Financieel plan}
In deze paragraaf wordt inzicht gegeven in de kosten (mensen, middelen en overig) van het project.
Aangegeven worden de resources die in de planning zijn opgenomen,
de hiervoor gehanteerde tarieven en de hieruit resulterende verwachte kosten.


\section{Kwaliteitsborging}
Dit hoofdstuk geeft inzicht in de relatie tussen de voorgestelde maatregelen
en de door de opdrachtgever gestelde eisen ten aanzien van de kwaliteit.
Hiernaast worden maatregelen getroffen om onderkende risico's uit te sluiten of de gevolgen te minimaliseren,
en de cruciale succesfactoren te be\"invloeden.

Als uitgangspunt worden de door de opdrachtgever gestelde kwaliteitseisen gehanteerd.
Deze worden verbijzonderd naar de te stellen kwaliteitseisen per product.
De voorgestelde maatregelen in het proces zijn een vertaling van deze vastgestelde productkwaliteitseisen.

Naast maatregelen in het proces om te voldoen aan de kwaliteitseisen per product worden additioneel maatregelen getroffen,
voor de kwaliteit van de tussenproducten of het proces zelf.
Laatstgenoemde wordt ontleend aan ondermeer de vereiste kwaliteit van besturing of het minimaliseren van risico's.

Alle maatregelen zijn in het proces ingebouwd en zijn dus elders in het plan van aanpak opgenomen als activiteit,
inrichtingsaspect of voorwaarde. Dit hoofdstuk geeft het totaaloverzicht van de invulling van het kwaliteitsaspect.

De paragrafen worden als volgt ingevuld:

\begin{description}
  \item[Productkwaliteit] Eisen per product per kwaliteitsattribuut voorzien van weging en acceptatiecriteria.
  Relatie met de gestelde eisen aan, en acceptatiecriteria van, het projectresultaat
  \item[Proceskwaliteit] Eisen te stellen aan het proces. Voorbeelden hiervan zijn:
  \begin{itemize}
    \item vakbekwaamheid
    \item gebruik van (systeem)ontwikkelmethode
    \item procedures
    \item gebruik van methode voor projectmanagement:
    \item uitbesteding en inkoop
  \end{itemize}

  Controle achteraf is mogelijk door verificatie en validatie

  \item[Voorgestelde maatregelen] Maatregelen in het proces met per maatregel de relatie naar de eisen.
  Voorbeelden hiervan zijn:
  \begin{itemize}
    \item opleidingsplan
    \item gebruik van methode voor systeemontwikkeling
    \item testplan
    \item gebruik van Managing Projects als methode voor projectmanagement
  \end{itemize}

  Maatregelen ter verificatie en validatie
  Voorbeelden hiervan zijn;
  \begin{itemize}
    \item audits
    \item reviews
  \end{itemize}

\end{description}

Bovenstaande, mogelijk lange en droge opsomming van,
relaties kunnen visueel meer inzichtelijk worden gemaakt door deze op te nemen in een matrix.


\section{Overige plannen}

In dit hoofdstuk worden alle plannen opgenomen die niet op tijd, geld en middelen zijn gericht.
De invulling is afhankelijk van de projectbehoefte.
Voorbeelden:
\begin{itemize}
  \item communicatieplan
  \item documentatieplan
  \item configuratiebeheerplan
  \item beveiligingsplan
\end{itemize}

\section{Bijlagen}
In dit hoofdstuk wordt verwezen naar de relevante standaards en projectprocedures.
In het voorkomend geval zal verwezen worden naar reeds bestaande c.q. gebruikelijke bedrijfsstandaards.
Voorwaarde is wel dat deze gedocumenteerd zijn.
In de bijlagen worden ook Begrippen en definities opgenomen om begripsverwarring te voorkomen.
De begrippenlijst hoeft niet uitputtend te zijn, alleen de gehanteerde begrippen in het Plan van Aanpak komen hiervoor in aanmerking.


%\newpage
%\bibliographystyle{plain}
%\bibliography{cassam}

\end{document}

