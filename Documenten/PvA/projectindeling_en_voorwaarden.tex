\section{Projectinrichting en voorwaarden}

Geef aan het begin van dit hoofdstuk een korte inleiding, want dat is tof. %Sjors: ehmm??

\subsection{Projectinrichting}

Het doel van projectinrichting is het zichtbaar maken van de wijze waarop de projectmanager,
van plan is het project in te richten om de opdracht uit te voeren volgens de voorgestelde aanpak.
Hierbij zal de gekozen inrichting afhankelijk zijn van de resultaten van de risico analyse,
kwaliteitseisen en de cruciale succesfactoren.

Afhankelijk van de opdracht en de organisatie komen de OPAFIT aspecten aan de orde:
\begin{description}
  \item[Organisatie] waarbij aangegeven wordt hoe de projectorganisatie eruit komt te zien inclusief taken en verantwoordelijkheden.
  Deze worden per persoon en per rol gesteld
  \item[Personeel] waarbij de eisen aan de gewenste inzet en beschikbaarheid van personeel worden aangegeven,
  zoals condities voor het betrekken van personeel, per groep de vereiste vakkennis, skills gerelateerd aan de plannen
  \item[Administratieve procedures] waarin alle binnen en rond het project van toepassing zijnde procedures worden genoemd
  \item[Financing] alle financi\"ele zaken worden hier behandeld, bij voorkeur met verwijzingen of, bij afwezigheid,
  expliciet opgenomen zoals tariefwijzigingen, facturering, subcontractors, btw en dergelijke;
  \item[Informatie] waarbij ingegaan wordt op alle informatie rond het project, overleg- en rapportagestructuren;
  \item[Techniek] waarbij wordt ingegaan op de voorgestelde inrichting qua hard- en software, werkplekken, hulpmiddelen en dergelijke.
\end{description}

\subsection{Voorwaarden aan opdrachtnemer}
Opsomming van voorwaarden, die gerealiseerd dienen te worden door de opdrachtnemer om het project volgens plan te kunnen uitvoeren.
Deze voorwaarden zijn gerelateerd aan en aanvullend op de inrichtingsaspecten.

\subsection{Voorwaarden aan opdrachtgever}

idem als 4.2, echter met opdrachtgever i.p.v. opdrachtnemer.

\subsection{Voorwaarden aan derden}

idem als 4.2, echter met derden i.p.v. opdrachtnemer.

