\section{Aanpak}
\label{aanpak_en_tijdsplanning}

We willen het resultaat bereiken door in verschillende fasen 

%In het hoofdstuk Aanpak wordt de brug geslagen tussen het afgebakende resultaat en de inrichting van het project, door middel van beantwoording van de ``hoe''-vraag.
%Doel is om door middel van Aanpak overeenstemming te verkrijgen over de te volgen weg, om te komen tot het gewenste resultaat.

%Per eindresultaat wordt aangegeven welke activiteiten zullen worden uitgevoerd en eventueel welke tussenresultaten worden opgeleverd.
%Tevens wordt hierbij ingegaan op het waarom van de gekozen oplossing.
%Daarbij wordt verwezen naar de cruciale succesfactoren, de resultaten van de uitgevoerde risico analyse, en de geformuleerde eisen en beperkingen ten aanzien van proces, resultaat en kwaliteit.
%Als de projectmanager daarin op basis van de uitgangspositie, cruciale succesfactoren, risico analyse of kwaliteitseisen onduidelijkheid of onvolledigheid vaststelt, geeft hij aan hoe hij met deze zaken omgaat.

%De projectmanager zal het project structureren en faseren, om aan te geven in welke globale stappen hij de projectopdracht denkt uit te voeren.

%Bij het structureren groepeert hij de gewenste eindresultaten primair naar algemene aandachtsgebieden.
%De volgende algemene aandachtsgebieden worden onderkend:

%- Ontwikkeling resultaat
%- Voorbereiding gebruik, dit zijn de activiteiten die samenhangen met het (her)inrichten van de gebruikersorganisatie
%- Voorbereiding beheer, dit zijn de activiteiten die samenhangen met het (her)inrichten van de beheerorganisatie
%- Acceptatie gebruik, het voorbereiden en uitvoeren van de gebruikers-acceptatie
%- Acceptatie beheer, het voorbereiden en uitvoeren van de beheeracceptatie
%- Kennis, dit zijn de activiteiten die samenhangen met het opbouwen van materiekennis met betrekking tot het resultaat (ook van het gebruik en het beheer ervan) en de activiteiten die samenhangen met de overdracht van deze kennis, naar de staande organisatie.

%Afhankelijk voor het type project worden de voor het project te hanteren aandachtsgebieden afgeleid uit de algemene aandachtsgebieden.
%Ook spelen andere criteria bij het structureren een rol, bijvoorbeeld:

%- risico factoren
%- cruciale succesfactoren
%- kwaliteitseisen

%Naast het structureren zal het project tevens in de tijd worden gefaseerd om formele meet- en beslismomenten te verkrijgen.
%De fasering wordt gericht op de beslissingen die de opdrachtgever wil nemen en vindt ondermeer plaats op basis van invoeringstijdstip of product.

%Per aandachtsgebied en verdere onderverdeling, wordt aangegeven door welke activiteiten het eindresultaat wordt bereikt, wat de samenhang van de activiteiten is en welke tussenresultaten worden opgeleverd binnen c.q. buiten de projectopdracht.
%Indien nodig kan de samenhang gevisualiseerd worden in de vorm van een eenvoudig netwerkplan zonder kwantitatieve gegevens.

%Conform de structuur en fasering wordt dit hoofdstuk in paragrafen opgedeeld.



\subsection{Risico analyse}

Het product maakt gebruik van meerdere software pakketten,
hier komt van nature een bepaald risico bij kijken in verband met het integreren in het product.
Op grond van een risicoanalyse kunnen de volgende maatregelen worden genomen:
\begin{description}
    \item[preventie] het voorkomen dat iets gebeurt of het verminderen van de kans dat het gebeurt;
    \item[repressie] het beperken van de schade wanneer een bedreiging optreedt;
    \item[acceptatie] geen maatregelen, men accepteert de kans en het mogelijke gevolg van een bedreiging;
    \item[manipulatie] het wijzigen van parameters in de berekening om tot een gewenst resultaat te komen.
\end{description}

De bedoeling van een risicoanalyse is dat er na de analyse wordt vastgesteld op welke wijze de risico's beheerst kunnen worden, of teruggebracht tot een aanvaardbaar niveau. \cite{wiki:risico_analyse}

We hebben de volgende risico factoren geidentificeerd:
\begin{description}
    \item[python] niet iedereen is bekend met python en heeft dus tijd nodig om op snelheid te komen
    \item[django] django is een template taal en vereist zeer weinig nieuwe kennis
    \item[database] het gebruik van een beveiligde database kan problemen veroorzaken
    \item[python image libraries] er zijn vanuit de Delft Computer Graphics group, niemand heeft deze eerder gebruikt
    \item[nose test] testen met een nieuw framework is altijd even wennen, en kan veel tijd kosten
\end{description}

Aan de hand van deze risico's hebben we de volgende milestone's opgesteld:
\begin{enumerate}
    \item Een eenvoudige applicatie die mbv Django een 'Hello world' pagina kan laten zien
    \item CRUD (Create, Read, Update, Delete) van data naar een database mbv Django
    \item Verwerken van images mbv python image libraries
    \item Tests voor iedere milestone
\end{enumerate}

\section{Plannen}
\label{plannen}

In dit hoofdstuk wordt verteld wat de algemene planning van het project is. 
De verschillende fases worden aangegeven, evenals de deadlines die voor die fases gelden.

%In het hoofdstuk plannen wordt de resultante vastgelegd van het evenwicht tussen activiteiten, tijd,
%geld en middelen teneinde de opdracht te kunnen uitvoeren.
%De verschillende paragrafen worden als volgt ingevuld:

\subsection{Normen en aannames}
Een van de door ons aangenomen normen is dat er voor het hele project gebruik wordt gemaakt van open source software, en dat het product van het project ook een open source product is. 
Daarnaast nemen we aan dat de algoritmes, geleverd door de TU Delft, correct en bewezen zijn.
Ook nemen we aan dat de foto's, geleverd door het \casamproject, onder gelijke omstandigheden zijn genomen en worden aangeleverd.
%Hierbij worden de gehanteerde normen, aannames en veronderstellingen zowel ten aanzien van de schattingen,
%als ten aanzien van planning vermeld, zoveel mogelijk per eenheid verbijzonderd.
%Deze kunnen afkomstig zijn uit geraadpleegde literatuur aangevuld met ``ervaringscijfers''.

\subsection{Activiteitenplan}
De opbouw van het product is opgedeeld in 3 fasen:
\begin{itemize}
	\item Database applicatie
	\item Active Shape Modeling
	\item Image Processing
\end{itemize}
Deze fasen worden hieronder per stuk verder toegelicht.

\subsubsection{Fase 1: Database applicatie}
In eerste instantie zal alle data die op dit moment in het \casamproject aanwezig is, gecentraliseerd worden en beheersbaar gemaakt d.m.v. een database en bijbehorende interface. 
Ook moet het mogelijk zijn om nieuwe data toe te voegen en in selecties van de data te zoeken. 
Ook het visualiseren van de resultaten is uiteraard van groot belang. 
Verder moet er een interface zijn voor de huidige onderzoeken en moet deze uitbreidbaar/aanpasbaar zijn voor nieuwe onderzoeken.

\subsubsection{Fase 2: Active Shape Modeling}
In het tweede deel van het project worden de foto's uit een van de projecten d.m.v. landmarks naar elkaar gewarped. 
Voor dit warpen maken we gebruik van verschillende al bestaande, uit wetenschappelijk onderzoek gebleken correcte, algoritmes.
Uiteraard moet het ook in deze fase mogelijk zijn voor een gebruiker om de foto's voor het warpen uit verschillende subsets van de data te halen.

\subsubsection{Fase 3: Image Processing}
In de laatste fase van het project, waarvan het niet zeker is dat we deze gaan halen, moet het mogelijk worden voor de onderzoekers om zelf landmarks aan te geven op r\"ontgenfoto's van de pati\"enten, om deze te warpen naar de reeds aanwezige foto's in de database. Deze fase zien wij als leuke toevoeging voor de dan bestaande presentatie, maar is geen vereiste voor het slagen van het project.
%In deze paragraaf worden de uit te voeren activiteiten beschreven.
%De detaillering hiervan is sterk afhankelijk van de opdrachtformulering en de fase waarin het project zich bevindt.
%Per activiteit wordt weergegeven de benodigde inspanning, de tijdsduur,
%de samenhang met andere activiteiten en het benodigde resourceniveau.

\subsection{Mijlpalen-/Productenplan}
De mijlpalen die gehaald moeten worden, staan aangegeven op de door ons gebruikte Trac server.
\begin{description}
	\item[Fase 0: Plan van Aanpak] Het plan van aanpak moet af zijn op 17 april.
	\item[Fase 1: Database applicatie] Fase 1 moet af zijn op 8 mei.
	\item[Fase 2: Active Shape Modeling] Fase 2 moet af zijn op 25 mei.
	\item[Fase 3: Image Processing] Fase 3 moet af zijn op 11 juni.
	\item[Fase 4: Eindverslag] Het eindverslag moet af zijn op 12 juni.
	\item[Fase 5: Eindpresentatie] De eindpresentatie moet gegeven zijn op 19 juni.
\end{description}
Het kan goed gebeuren dat de deadline van fase 3 niet gehaald wordt.
Een precieze datum voor de eindpresentatie wordt t.z.t. bepaald.
%Het mijlpalenplan geeft de meet- of beslismomenten weer.
%Hierbij worden de meest belangrijke momenten voor toetsing en sturing benadrukt.
%Het productenplan geeft de momenten weer waarop de (tussen)producten zullen worden opgeleverd en geaccepteerd.

\subsection{Resourceplan}
De studenten van de TU Delft die aan dit project werken, zijn:
\begin{itemize}
	\item Sjors van Berkel
	\item Bastiaan Bijl
	\item Jaap den Hollander
	\item Sverre Rabbelier
	\item Ben Sedee
	\item Noeska Smit
\end{itemize}
Al deze personen zitten in hun derde studiejaar van de opleiding Technische Informatica, en zitten dus op hetzelfde niveau. 
De overige personen die hun medewerking verlenen aan dit project zijn:
\begin{description}
	\item[Dr. Botha] Begeleider van de TU Delft
	\item[Anton Kerver] Student aan het EMC, nauw betrokken bij het \casamproject
	\item[Dr. Kleinrensink] projectleider van het \casamproject
\end{description}
De niet personele middelen waar wij gebruik van maken zijn:
\begin{itemize}
	\item SVN Server van de TU Delft
	\item Trac Server van de TU Delft
	\item Django Server van de TU Delft
	\item Eclipse in combinatie met de PyDev plugin
\end{itemize}
Alle voor dit project gebruikte software, is vrij gegeven onder een open source licentie.
Het gehele project begint op 6 april 2009, en eindigt op 19 juni 2009. 
Gedurende deze periode zal er full-time (40 uur per week) gewerkt worden aan het project.

%Het resourceplan verschaft duidelijkheid over personele en overige middelen.
%Het plan geeft weer over welke perioden inzet benodigd is. Bij de personele middelen wordt tevens het niveau van de resource %aangegeven.

\subsection{Financieel plan}
Aangezien dit project een bachelorproject is van studenten van de TU Delft, en alle resources die benodigd zijn worden geleverd door de TU Delft en/of het \casamproject, zijn er met dit project geen kosten gemoeid. Er staat waarschijnlijk ook geen vergoeding tegenover.
%In deze paragraaf wordt inzicht gegeven in de kosten (mensen, middelen en overig) van het project.
%Aangegeven worden de resources die in de planning zijn opgenomen,
%de hiervoor gehanteerde tarieven en de hieruit resulterende verwachte kosten.\


Vanwege de duur van het project, en onze verwachting dat we fase 3 niet volledig zullen afronden. Hebben wij geen plannen voor als er ruimte overblijft aan het eind van project. Uiteraard zal er in het eindverslag wel aandacht besteed worden aan de mogelijke vervolgtrajecten na dit project.
