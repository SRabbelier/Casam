\section{Aanpak en tijdsplanning}
\label{aanpak_en_tijdsplanning}
%OMSCHRIJVING SECTIE
%In dit hoofdstuk geeft de projectmanager aan hoe hij denkt te kunnen voldoen aan de in Sectie 2.5 gestelde eisen. Hierbij wordt de fasering van het project besproken en wordt aangegeven van welke aannames wordt uitgegaan.

In dit hoofdstuk geven we aan hoe we aan de gestelde eisen en verwachtingen denken te kunnen voldoen. We werken hier een fasering van de aanpak uit en zullen activiteiten opsommen. Er komt ook een tijdsplanning.

\subsection{Fasering}

De opbouw van het product is opgedeeld in 3 productiefasen:
\begin{itemize}
	\item Database applicatie
	\item Active Shape Modeling
	\item Image Processing
\end{itemize}
Deze fasen worden hieronder per stuk verder toegelicht.

\subsubsection{Fase 1: Database applicatie}
In eerste instantie zal alle data die op dit moment in het \casamproject aanwezig is, gecentraliseerd worden en beheersbaar gemaakt d.m.v. een database en bijbehorende interface. 
Ook moet het mogelijk zijn om nieuwe data toe te voegen en in selecties van de data te zoeken. 
Ook het visualiseren van de resultaten is uiteraard van groot belang. 
Verder moet er een interface zijn voor de huidige onderzoeken en moet deze uitbreidbaar/aanpasbaar zijn voor nieuwe onderzoeken.

\subsubsection{Fase 2: Active Shape Modeling}
In het tweede deel van het project worden de foto's uit een van de projecten d.m.v. landmarks naar elkaar gewarped. 
Voor dit warpen maken we gebruik van verschillende al bestaande, uit wetenschappelijk onderzoek gebleken correcte, algoritmes.
Uiteraard moet het ook in deze fase mogelijk zijn voor een gebruiker om de foto's voor het warpen uit verschillende subsets van de data te halen.

\subsubsection{Fase 3: Image Processing}
In de laatste fase van het project, waarvan het niet zeker is dat we deze gaan halen, moet het mogelijk worden voor de onderzoekers om zelf landmarks aan te geven op r\"ontgenfoto's van de pati\"enten, om deze te warpen naar de reeds aanwezige foto's in de database. Deze fase zien wij als leuke toevoeging voor de dan bestaande presentatie, maar is geen vereiste voor het slagen van het project.

\subsection{Tijdsplanning}
%OMSCHRIJVING SECTIE
%Vervolgens wordt een lijst met activiteiten opgesteld en wordt een schatting gemaakt van de nodige resources en tijd voor deze activiteiten. Op grond hiervan wordt een tijdsplanning gegeven met bijbehorende mijlpalen. Hierbij wordt ook aangegeven op welke momenten input van de opdrachtgever nodig is en wanneer terugkoppeling plaatsvindt.




De mijlpalen die gehaald moeten worden, staan aangegeven op de door ons gebruikte Trac server.
\begin{description}
	\item[Fase 0: Plan van Aanpak] Het plan van aanpak moet af zijn op 17 april.
	\item[Fase 1: Database applicatie] Fase 1 moet af zijn op 8 mei.
	\item[Fase 2: Active Shape Modeling] Fase 2 moet af zijn op 25 mei.
	\item[Fase 3: Image Processing] Fase 3 moet af zijn op 11 juni.
	\item[Fase 4: Eindverslag] Het eindverslag moet af zijn op 12 juni.
	\item[Fase 5: Eindpresentatie] De eindpresentatie moet gegeven zijn op 19 juni.
\end{description}
Het kan goed gebeuren dat de deadline van fase 3 niet gehaald wordt.
Een precieze datum voor de eindpresentatie wordt t.z.t. bepaald.

Vanwege de duur van het project, en onze verwachting dat we fase 3 niet volledig zullen afronden. Hebben wij geen plannen voor als er ruimte overblijft aan het eind van project. Uiteraard zal er in het eindverslag wel aandacht besteed worden aan de mogelijke vervolgtrajecten na dit project.
