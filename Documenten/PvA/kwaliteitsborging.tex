\section{Kwaliteitsborging}
\label{kwaliteit}

Dit hoofdstuk geeft inzicht in de relatie tussen de voorgestelde maatregelen
en de door de opdrachtgever gestelde eisen ten aanzien van de kwaliteit.
Hiernaast worden maatregelen getroffen om onderkende risico's uit te sluiten of de gevolgen te minimaliseren,
en de cruciale succesfactoren te be\"invloeden.

Als uitgangspunt worden de door de opdrachtgever gestelde kwaliteitseisen gehanteerd.
Deze worden verbijzonderd naar de te stellen kwaliteitseisen per product.
De voorgestelde maatregelen in het proces zijn een vertaling van deze vastgestelde productkwaliteitseisen.

Naast maatregelen in het proces om te voldoen aan de kwaliteitseisen per product worden additioneel maatregelen getroffen,
voor de kwaliteit van de tussenproducten of het proces zelf.
Laatstgenoemde wordt ontleend aan ondermeer de vereiste kwaliteit van besturing of het minimaliseren van risico's.

Alle maatregelen zijn in het proces ingebouwd en zijn dus elders in het plan van aanpak opgenomen als activiteit,
inrichtingsaspect of voorwaarde. Dit hoofdstuk geeft het totaaloverzicht van de invulling van het kwaliteitsaspect.

De paragrafen worden als volgt ingevuld:

\begin{description}
  \item[Productkwaliteit] Eisen per product per kwaliteitsattribuut voorzien van weging en acceptatiecriteria.
  Relatie met de gestelde eisen aan, en acceptatiecriteria van, het projectresultaat
  \item[Proceskwaliteit] Eisen te stellen aan het proces. Voorbeelden hiervan zijn:
  \begin{itemize}
    \item vakbekwaamheid
    \item gebruik van (systeem)ontwikkelmethode
    \item procedures
    \item gebruik van methode voor projectmanagement:
    \item uitbesteding en inkoop
  \end{itemize}

  Controle achteraf is mogelijk door verificatie en validatie

  \item[Voorgestelde maatregelen] Maatregelen in het proces met per maatregel de relatie naar de eisen.
  Voorbeelden hiervan zijn:
  \begin{itemize}
    \item opleidingsplan
    \item gebruik van methode voor systeemontwikkeling
    \item testplan
    \item gebruik van Managing Projects als methode voor projectmanagement
  \end{itemize}

  Maatregelen ter verificatie en validatie
  Voorbeelden hiervan zijn;
  \begin{itemize}
    \item audits
    \item reviews
  \end{itemize}

\end{description}

Bovenstaande, mogelijk lange en droge opsomming van,
relaties kunnen visueel meer inzichtelijk worden gemaakt door deze op te nemen in een matrix.

