\section{Kwaliteitsborging}
\label{kwaliteit}
In dit hoofdstuk wordt een kort overzicht gegeven van de kwaliteitseisen die aan het product en het proces worden gesteld,
en worden een aantal maatregelen voorgesteld om aan deze eisen te kunnen voldoen.
%Dit hoofdstuk geeft inzicht in de relatie tussen de voorgestelde maatregelen
%en de door de opdrachtgever gestelde eisen ten aanzien van de kwaliteit.
%Hiernaast worden maatregelen getroffen om onderkende risico's uit te sluiten of de gevolgen te minimaliseren,
%en de cruciale succesfactoren te be\"invloeden.

%Als uitgangspunt worden de door de opdrachtgever gestelde kwaliteitseisen gehanteerd.
%Deze worden verbijzonderd naar de te stellen kwaliteitseisen per product.
%De voorgestelde maatregelen in het proces zijn een vertaling van deze vastgestelde productkwaliteitseisen.

%Naast maatregelen in het proces om te voldoen aan de kwaliteitseisen per product worden additioneel maatregelen getroffen,
%voor de kwaliteit van de tussenproducten of het proces zelf.
%Laatstgenoemde wordt ontleend aan ondermeer de vereiste kwaliteit van besturing of het minimaliseren van risico's.

%Alle maatregelen zijn in het proces ingebouwd en zijn dus elders in het plan van aanpak opgenomen als activiteit,
%inrichtingsaspect of voorwaarde. Dit hoofdstuk geeft het totaaloverzicht van de invulling van het kwaliteitsaspect.

%De paragrafen worden als volgt ingevuld:

%\begin{description}
%  \item[Productkwaliteit] Eisen per product per kwaliteitsattribuut voorzien van weging en acceptatiecriteria.
%  Relatie met de gestelde eisen aan, en acceptatiecriteria van, het projectresultaat

\subsection{Productkwaliteit}
Om de kwaliteit van het product te garanderen worden een aantal eisen gesteld waar het eindproduct aan zal moeten voldoen.
Er zijn eisen door de opdrachtgever gegeven om te zorgen dat het product daadwerkelijk aan de gebruikerseisen voldoet.
Zo is vastgesteld dat het systeem het niet mogelijk maakt dat de gebruikte data uitlekt naar een ongeautoriseerde gebruiker,
analyse van de gebruikte afbeeldingen moet wiskundig kunnen worden zodat de uitslag hiervan niet betwist kan worden.
Bovendien moet het product simpel genoeg worden om makkelijk mee te kunnen werken,
en om automatisch de benodigde orde aan te brengen in de aangemaakte data, zodat deze uniform en uitwisselbaar wordt en blijft.
Andere kwaliteitseisen aan het product zijn van meer technische aard, en zullen worden gespecificeerd in het requirements document. %Als er uberhaupt een requirements document komt
De kwaliteit van het product wordt zowel door kwaliteitsverzekering als door kwaliteitscontrole veiliggesteld.

%  \item[Proceskwaliteit] Eisen te stellen aan het proces. Voorbeelden hiervan zijn:
%  \begin{itemize}
%    \item vakbekwaamheid
%    \item gebruik van (systeem)ontwikkelmethode
%    \item procedures
%    \item gebruik van methode voor projectmanagement:
%    \item uitbesteding en inkoop
%  \end{itemize}

\subsection{Proceskwaliteit}
De kwaliteit van het proces wordt bereikt door de vakbekwaamheid van de projectuitvoerders,
de ervaringen in de gebruikte software engineering methode,
en het zorgvuldig vaststellen van de gebruikte methoden en componenten.
Verder wordt door regelmatig contact met de opdrachtgever, en een duidelijke planning, 
het proces gericht gehouden op een tijdige aflevering van een product dat voldoet aan de eisen.

\subsection{Voorgestelde maatregelen}
Om bovenstaande kwaliteitsnormen te halen worden de volgende maatregelen voorgesteld
\begin{itemize}
\item dagelijks kort overleg tussen de uitvoerders, en wekelijks een uitgebreid overleg
\item wekelijkse review van het proces en de software met de opdrachtgever
\item unit testing van de software
\end{itemize}

%  Controle achteraf is mogelijk door verificatie en validatie

%  \item[Voorgestelde maatregelen] Maatregelen in het proces met per maatregel de relatie naar de eisen.
%  Voorbeelden hiervan zijn:
%  \begin{itemize}
%    \item opleidingsplan
%    \item gebruik van methode voor systeemontwikkeling
%    \item testplan
%    \item gebruik van Managing Projects als methode voor projectmanagement
%  \end{itemize}

%  Maatregelen ter verificatie en validatie
%  Voorbeelden hiervan zijn;
%  \begin{itemize}
%    \item audits
%    \item reviews
%  \end{itemize}

%\end{description}

%Bovenstaande, mogelijk lange en droge opsomming van,
%relaties kunnen visueel meer inzichtelijk worden gemaakt door deze op te nemen in een matrix.

