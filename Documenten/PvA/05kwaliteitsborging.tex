\section{Kwaliteitsborging}
\label{kwaliteit}
%%OMSCHRIJVING SECTIE
%In het laatste hoofdstuk wordt uitgewerkt op welke wijze de opdrachtgever en de projectleden in staat zijn de kwaliteit van het project te be�nvloeden. Hoewel harde voorwaarden in hoofdstuk 2 zijn verwoord, wordt hier verder uitgewerkt wat met name de opdrachtgever kan doen om bij te dragen aan een hoge kwaliteit.
Als laatste worden de maatregelen beschreven die beide partijen zullen treffen om bekende risico's te voorkomen.

In dit hoofdstuk wordt een kort overzicht gegeven van de kwaliteitseisen die aan het product en het proces worden gesteld,
en worden een aantal maatregelen voorgesteld om aan deze eisen te kunnen voldoen.

\subsection{Productkwaliteit}
Om de kwaliteit van het product te garanderen worden een aantal eisen gesteld waar het eindproduct aan zal moeten voldoen.
Er zijn eisen door de opdrachtgever gegeven om te zorgen dat het product daadwerkelijk aan de gebruikerseisen voldoet.
Zo is vastgesteld dat het systeem zoveel mogelijk moet doen om te voorkomen dat de gebruikte data niet uitlekt naar een ongeautoriseerde gebruiker. En dat de warping-technologie die gebruikt wordt wiskundig onderbouwd en bewezen is.
Daarnaast moet het product simpel in de omgang zijn en moet het weinig tijd kosten om er mee bekend te raken.
Ook moet er automatisch orde worden aangebracht de aangeleverde data, zodat deze uniform en uitwisselbaar wordt en blijft.
Andere kwaliteitseisen aan het product zijn van meer technische aard, en zijn gespecificeerd in het requirements document.
De kwaliteit van het product wordt zowel door kwaliteitsverzekering als door kwaliteitscontrole veiliggesteld.

\subsection{Proceskwaliteit}
De kwaliteit van het proces wordt bereikt door de vakbekwaamheid van de projectuitvoerders,
de ervaringen in de gebruikte software engineering methode en het zorgvuldig vaststellen van de gebruikte methoden en componenten.
Bovendien wordt gebruikt gemaakt van agile programmeren waardoor er regelmatig contact met de opdrachtgever is en het proces zo tijdig kan worden bijgestuurd als dit nodig is. Aan het eind onstaat dan vanzelf een product van hoge kwaliteit.

\subsection{Voorgestelde maatregelen}
Om bovenstaande kwaliteitsnormen te halen worden de volgende maatregelen voorgesteld
\begin{itemize}
\item dagelijks kort overleg tussen de uitvoerders, en wekelijks een uitgebreid overleg met de opdrachtgever
\item wekelijkse review van het proces en de software met de opdrachtgever
\item unit testing van de software
\end{itemize}
