\section{Projectopdracht}
\label{projectopdracht}

%OMSCHRIJVING SECTIE
%In dit hoofdstuk wordt de gewenste verandering in beeld gebracht. De opdracht wordt afgebakend en vastgelegd zodat opdrachtgever en de projectmanager helder hebben waaraan het product zal moeten voldoen. Het heeft het karakter van een contract.

In dit hoofdstuk zal de gewenste verandering ten opzichte van de huidige situatie in beeld gebracht worden. 
Deze verandering zal in `opdrachtgevers bewoording' aan de orde worden gebracht. 


\subsection{Projectomgeving}
%OMSCHRIJVING SECTIE
%Omschrijving van de activiteiten en processen van de opdrachtgever die leidden tot de probleemstelling. Omschrijving dus van de huidige (problematische) situatie. 

De projectomgeving waarbinnen we voor het \casamproject gaan werken bestaat uit de onderzoeksafdeling Neurowetenschappen-Anatomie van het Erasmus Medisch Centrum, in samenwerking met de vakgroep Computer Graphics, specialisatie Medische Visualisatie, van de Technische Universiteit Delft. 
Op dit moment worden er bij het Erasmus Medisch Centrum verschillende onderzoeken gedaan binnen het \casamproject. 
\casam is een methode om van foto's van verschillende anatomische preparaten van eenzelfde structuur, waarop voor het onderzoek relevante structuren (zenuwen, venen, spiergroepen, etc.) worden gemarkeerd, \'e\'en gemiddelde te berekenen. 
In de huidige sitatie worden de meetgegevens en foto's `met de hand' opgeslagen. 
Ook gebeurt het `morphen' van de foto's naar de gemiddelde foto voornamelijk handmatig en met `out-of-the-box' software. 
Dit handmatig `morphen' is tijdrovend. 
Een ander belangrijk probleem is dat er geen wetenschappelijk verantwoorde methode wordt gebruikt voor het `morphen' van de foto's.


\subsection{Doelstelling project}
%OMSCHRIJVING SECTIE
%Waarom heeft de opdrachtgever het resultaat nodig en wat wil de opdrachtgever met het resultaat bereiken? Iedere doelstelling wordt zo mogelijk uitgedrukt in kwalitatieve en kwantitatieve verwoordingen.

De \casamproject-groep heeft het resultaat van de opdracht nodig om op een wetenschappelijk verantwoorde manier onderzoek te kunnen doen naar anatomische vraagstukken. 
Het invoeren van de meetresultaten moet gemakkelijker en effici\"enter worden. 
Ook is het resultaat van de opdracht zeer belangrijk als een manier om de verworven kennis over te brengen op chirurgen, zodat het ook in de operatiekamer kan worden toegepast. 
Bij het huidige anatomische onderzoek wordt nu gebruik gemaakt van tabellen, illustraties of voorbeeldfoto's om de resultaten over te brengen.
Voor een chirurg is geen van bovenstaande weergavemogelijkheden echter direct praktisch bruikbaar. Dit omdat deze niet duidelijk en precies genoeg zijn en omdat variaties niet goed worden weegegeven.



\subsection{Opdrachtformulering}
%OMSCHRIJVING SECTIE
%Wat is de projectopdracht? Welke zaken worden wel en welke niet tot de verantwoordelijkheid van het project gerekend? Er wordt vastgelegd wat er van de projectgroep verwacht wordt, hoe de opdrachtgever verwacht dat er te werk wordt gegaan.

De volgende stap in het \casamproject moet worden gemaakt, wat voor de opdrachtgever betekent dat er een wetenschappelijk verantwoorde methode voor de beeldvorming moet worden ontwikkeld. 
Om dit te kunnen doen moet het morph-proces geverifieerd worden. 
Ook moeten andere en eventueel betere methoden zoals Active Shape Modelling worden bekeken en onderling vergeleken. 
Verder is het van belang de meetresultaten en foto's makkelijk te kunnen opslaan, opvragen en wijzigen. 


\subsection{Op te leveren producten en diensten}
%OMSCHRIJVING SECTIE
%Wat is het resultaat van het project? Wat verwacht de opdrachtgever inhoudelijk van het product?

Het resultaat van het project is op zijn minst een overzichtelijke systeem, waarbij het invoeren en opvragen van data zal worden vergemakkelijkt.
Dit systeem zal dienen voor het opslaan van metingen en foto's, waarbij analyses per been, per subgroep, voor verschillende landmarks opvraagbaar moeten zijn.
Ook het visualiseren van de resultaten is uiteraard van groot belang.
Verder moet er een interface zijn voor de huidige onderzoeken en moet deze uitbreidbaar en aanpasbaar zijn voor nieuwe onderzoeken. 
\\
In de tweede fase van het project hopen we een tool te leveren die met behulp van o.a. Active Shape Modeling foto's kan morphen en de variaties in kaart kan brengen.
Hierbij moet het, gegeven een aantal corresponderende punten over alle datasets, mogelijk zijn de hoodmodi van de variatie van de punten over alle datasets uit te rekenen.
Het morphen moet Landmark-based plaatsvinden met gevestigde en wiskundig bewezen morphtechnieken. Met bijvoorbeeld een thin-plate spline transformatie moet het mogelijk zijn om met behulp van de punten en het shape model te morphen tussen de verschillende configuraties en tussen individu\"ele onderzoeken.
\\
Als er nog tijd beschikbaar is hopen we een tool te cre\"eren waarbij het mogelijk zal zijn voor een arts om bony landmarks op een r\"ontgenfoto aan te geven, en aan de hand daarvan een gemiddelde anatomie uit de database eroverheen te projecteren als begeleiding bij bijvoorbeeld een operatie(planning).

\subsection{Eisen en beperkingen}

%OMSCHRIJVING SECTIE
%Tenslotte worden harde eisen gesteld aan het product en de wijze waarop het product tot stand komt. De eisen moeten zo nauwkeurig mogelijk worden gekwantificeerd. Er worden ook al prioriteiten vastgesteld.

Het systeem moet zeker voldoen aan de volgende eisen:
\begin{itemize}
  \item Een onderzoeker moet projecten kunnen opvragen (gedeeltelijk of geheel), toevoegen en wijzigen
  \item Zowel een chirurg als een onderzoeker moet resultaten kunnen opvragen
  \item De gevoelige data moet uitsluitend beschikbaar zijn aan mensen met de juiste bevoegdheden.
\end{itemize}
Hierbij is met name de privacy een erg belangrijk punt.
De gegevens zijn erg gevoellig en moeten zo goed mogelijk beschermd worden.
Het programma moet om die reden ook draaien op een server binnen het Erasmus MC.

\subsection{Cruciale succesfactoren}
%OMSCHRIJVING SECTIE
%Hier wordt beschreven waaraan de opdrachtgever en de projectleden moeten voldoen om de gestelde eisen waar te kunnen maken. De verantwoordelijkheden van de opdrachtgever worden dus hier vastgelegd, alsmede die van de projectmanager en derden.

Het is absoluut cruciaal dat er aan de volgende eisen wordt voldaan:
\begin{itemize}
    \item Snelheid (arts moet binnen een minuut al resultaat kunnen zien, anders kan zijn tijd nuttiger besteed worden)
    \item Gemakkelijk aan te leren (intu\"itief)
    \item Duidelijke interface
\end{itemize}

\pagebreak