\section{Projectopdracht}
\label{projectopdracht}
In dit hoofdstuk zal de gewenste verandering ten opzichte van de huidige situatie in beeld gebracht. Deze verandering zal in de 'opdrachtgevers bewoording' aan de orde worden gebracht. 

%In dit hoofdstuk wordt de gewenste verandering in beeld gebracht.
%De opdracht wordt afgebakend, door middel van het beantwoorden van de %``waarom'', de ``waarover'' en de ``wat''-vragen.

%Deze zaken worden in ``opdrachtgevers bewoordingen'' aan de orde gebracht.
%De paragrafen worden als volgt ingevuld:

\subsection{Projectomgeving}
De projectomgeving waarbinnen we voor het aan het \casamproject gaan werken bestaat uit de onderzoeksafdeling Neurowetenschappen-Anatomie van het Erasmus MC in samenwerking met de vakgroep Computer Graphics, specialisatie Medische Visualisatie van de TU Delft. Op dit moment worden er bij het Erasmus MC verschillende onderzoeken gedaan in de \casam werkgroep. \casam is een methode om van foto's van meerdere anatomische preparaten van eenzelfde structuur, waarop voor het onderzoek relevante structuren (zenuwen, venen, spiergroepen, etc.) worden gemarkeerd, ��n gemiddelde te berekenen. In de huidige sitatie worden de meetgegevens en foto's 'met de hand' opgeslagen. Ook gebeurt het 'warpen' van de foto's naar de gemiddelde foto voornamelijk handmatig en met out-of-the-box software. Dit zorgt ervoor dat het tijdrovend is om de resultaten te verkrijgen. Een ander belangrijk probleem is dat er geen wetenschappelijk verantwoorde methode wordt gebruikt om te 'warpen'.

%Wat is het beschouwingsgebied?
%Hierin wordt een schets gegeven van het beschouwingsgebied in termen van %organisatie eenheden en bedrijfsprocessen.
%Tevens wordt aangegeven wat de problemen en oorzaken zijn die aanleiding geven tot de ontwikkeling van het resultaat.

\subsection{Doelstelling project}
De /casamproject heeft het resultaat van de opdracht nodig om op een wetenschappelijk verantwoorde manier onderzoek te kunnen doen naar anatomische vraagstukken. Het moet invoeren van de meetresultaten moet gemakkelijker en efficienter worden. Ook is het resultaat van de opdracht zeer belangrijk als een manier om de verworven kennis over te brengen op chirurgen zodat het ook in de operatiekamer kan worden toegepast. Bij het huidige anatomische onderzoek wordt gebruik gemaakt van tabellen, illustraties of een voorbeeldfoto om de resultaten over te brengen. Voor een chirurg echter is geen van bovenstaande weergavemogelijkheden direct en praktisch bruikbaar.

%Waarom heeft de opdrachtgever het resultaat nodig en wat wil de opdrachtgever met het resultaat bereiken?
%In deze paragraaf wordt een beschrijving gegeven van de doelstellingen van het te ontwikkelen resultaat,
%zoals aangegeven door de opdrachtgever.
%Met name wordt hierbij de koppeling gelegd naar bedrijfsprocessen.
%Hierbij is het van belang om te weten, waarop de opdrachtgever wordt afgerekend.
%Iedere doelstelling wordt zo mogelijk onderbouwd door kwalitatieve en kwantitatieve gegevens.

\subsection{Opdrachtformulering}
De volgende stap in het \casamproject moet worden gemaakt, dit houdt voor de opdrachtgever in dat er een wetenschappelijk verantwoorde methode voor de beeldvorming moet worden ontwikkeld. Om dit te kunnen doen moet het warp-proces geverifieerd worden. Ook moeten andere en eventueel betere methoden zoals Active Shape Modelling worden bekeken en onderling vergeleken. Verder is het van belang de meetresultaten en foto's makkelijk te kunnen opslaan en opvragen. 
%Wat is de projectopdracht?
%Waarover gaat het project procesmatig (afbakening)?
%Deze paragraaf beschrijft de opdracht, voortvloeiend uit de doelstelling, zoals aangegeven door de opdrachtgever.
%Hierbij wordt expliciet aangegeven welke zaken wel en welke zaken niet tot de verantwoordelijkheid van het project worden gerekend.
%Aangegeven wordt ook of het een resultaat- of een inspanningsverplichting betreft.

\subsection{Op te leveren producten en diensten}
Het resultaat van het project is op zijn minst een overzichtelijke database, waarbij het invoeren en opvragen van data zal worden vergemakkelijkt. Deze database zal dienen voor het opslaan van metingen en foto’s, hierbij moeten analyses per been, per subgroep, voor verschillende landmarks opvraagbaar zijn. Ook het visualiseren van de resultaten is uiteraard van groot belang. Verder moet er een interface zijn voor de huidige onderzoeken en moet deze uitbreidbaar/aanpasbaar zijn voor nieuwe onderzoeken. 

In de tweede fase van het project hopen we een tool te leveren die met behulp van o.a. Active Shape Modeling kan morphen en de variaties in kaart kan brengen. Hierbij moet het mogelijk zijn te varieren over hoofdmodi van populatie. Het morphen moet Landmark-based plaatsvinden met gevestigde warptechnieken.

Als er nog tijd beschikbaar is hopen we een tool te creeeren waarbij het mogelijk zal zijn voor een arts om bony landmarks op een röntgenfoto aan te geven en aan de hand daarvan een gemiddelde anatomie uit de database eroverheen te projecteren als geleide. 
%Wat is het resultaat van het project?
%Waarover gaat het project inhoudelijk (afbakening)?
%Deze paragraaf bevat de specificatie van de op te resultaten zoals aangegeven door de opdrachtgever.
%Dit is een nadere uitwerking van de projectopdracht, zoals aangegeven bij de opdrachtformulering.

\subsection{Eisen en beperkingen}
Het systeem moet zeker voldoen aan de volgende eisen:
\begin{itemize}
    \item Privacy moet gewaarborgd worden
    \item meer eisen hier.....
\end{itemize}
Hierbij is met name de privacy een erg belangrijk punt. De gegevens zijn erg gevoellig en moeten zo goed mogelijk beschermd worden.
%In deze paragraaf worden de acceptatiecriteria en beperkingen vermeld, die de opdrachtgever stelt aan het resultaat en de eisen en beperkingen die gesteld worden aan de gebruikte resources en aan de wijze, waarop het resultaat tot stand komt.
%De eisen moeten zo nauwkeurig mogelijk worden gekwantificeerd.
%Indien mogelijk worden er ook prioriteiten vastgesteld.

\subsection{Cruciale succesfactoren}
Het is absoluut cruciaal dat er aan de volgende eisen:
\begin{itemize}
    \item Snelheid (arts moet binnen minuut al resultaat kunnen zien, anders kan zijn tijd nuttiger besteed worden)
    \item Gemakkelijk aan te leren (intuïtief)
    \item Duidelijke interface
\end{itemize}

%Deze paragraaf beschrijft de door de opdrachtgever onderkende en specifiek voor deze opdracht geldende cruciale succesfactoren.
%Het moet zowel de opdrachtgever als de projectmanager duidelijk zijn welke maatregelen mogelijk zijn, c.q. door beiden genomen moeten worden om deze factoren te be\"invloeden.

%Van groot belang is de juiste interpretatie van een aantal onderdelen van de Projectopdracht:

%De Doelstelling geeft aan wat het achterliggende doel is van het starten van het project.
%Dit kan het doorvoeren van een organisatorische verandering zijn op uiteenlopende niveau's, zoals klant-, bedrijfs-, effici\"entie-, of middelenniveau.

%De Opdrachtformulering geeft weer door welk middel de opdrachtgever de gewenste doelstelling denkt te bereiken.

%De Eisen en beperkingen geven aan welke eisen de opdrachtgever stelt aan het eindresultaat en het procesmatige verloop van de opdracht.

%De Cruciale Succesfactoren geven aan,  welke door de opdrachtnemer be\"invloedbare zaken er, vanuit de opdrachtgever gezien, essentieel zijn om het resultaat zo goed mogelijk te laten aansluiten bij de te bereiken doelstelling.
