%\setcounter{section}{-1}
\section{Samenvatting}

Met het \casamproject is er tien weken lang gewerkt door zes Bachelor studenten van de TU Delft aan een 
systeem dat het \casamproject op het Erasums Medisch Centrum faciliteert. 
In dit ziekenhuis is Anton Kerver bezig meerdere preparaten te maken om voor bepaalde chirurgische ingrepen visueel te kunnen tonen hoe een gemiddelde pati\"{e}nt in elkaar zit.
\\
\\
Het systeem dat ontworpen is biedt een databasestructuur met een browser-interface waarin foto's van 
preparaten behorend tot \'{e}\'{e}n ingreep als project gegroepeerd opgeslagen worden. 
Vervolgens kunnen er per foto landmarks aangegeven worden en kunnen belangrijke gebieden overgetekend worden.
Ook is het mogelijk om meerdere foto's zichtbaar te maken, in combinatie met de landmarks en bitmaps die bij deze foto's horen.
Deze landmarks en bitmaps kunnen vervolgens ook eventueel onzichtbaar gemaakt worden.
Een gemaakte compositie van zichtbare foto's, landmarks en bitmaps kan opgeslagen worden als state, zodat deze later weer makkelijk op te vragen is.
\\
\\
Als op meerdere foto's dezelfde landmarks zijn aangegeven, is het mogelijk om visueel de variatie te tonen binnen de landmarks met behulp van een Point Distribution Model. 
Op deze manier kunnen verschillen in anatomie gemakkelijk inzichtelijk gemaakt worden.
Een tweede toepassing van Image Processing is de mogelijkheid om verschillende bitmaps met elkaar te morphen. 
Dit houdt in dat de verschillende bitmaps zodanig vervormd en verplaatst worden, dat ze allemaal op hetzelfde been van toepassing zijn,
zodat de variatie tussen bepaalde gebieden zichtbaar gemaakt kan worden.
\\
\\
Het is verder mogelijk om papers en weblinks aan projecten te koppelen.
Hiernaast is het systeem voorzien van een aantal hulpfuncties zoals een zoomscherm en de mogelijkheid om gedane acties ongedaan te maken. 
\\
\\
Bij met maken van het systeem is gebruik gemaakt van een Django framework dat met Python werkt en webpagina's genereert die op de client een JavaScript-applicatie draait. 
Ook is er voor \'{e}\'{e}n functie een Flash-applicatie geschreven.
Deze samensmelting van verschillende technologi\"{e}n was voor onze groep een volledig nieuwe ervaring, die 
ons uiteindelijk goed is bevallen.
