%\setcounter{section}{-1}
\section{Samenvatting}

Met het CASAM-project is er een kwartaal gewerkt door 6 Bachelorstudenten van de TU-Delft aan een systeem dat het CASAM-project op het Erasums MC faciliteerd. Op dit ziekenhuis is Anton Kerver bezig meerdere preparaten te maken voor bepaalde chirurgische ingrepen om visueel te kunnen tonen hoe geen gemiddelde pati\"ent in elkaar zit.

Het systeem dat ontworpen is biedt een databasestructuur met een browser-interface waarin preparaten behorend tot \'e\'en ingreep als project gegroepeerd opgeslagen kunnen worden. Vervolgens kunnen per foto landmarks aangegeven worden en kunnen belangrijke gebieden overgetekend worden.

Als op meerdere prepareten dezelfde landmarks zijn aangegeven is het mogelijk om visueel de spreiding te tonen van de landmarks. Dit is handig om verschillen tussen verschillende mensen inzichtelijk te maken. Een tweede toepassing van imageprocessing is het mogelijkheid om verschillende foto's naar elkaar te morphen. Dit houdt in dat beide dezelfde afmetingen krijgen en dat de afbeeldingen geschaald worden op een manier die de prepraten netjes over elkaar toont.

Het is mogelijk om meerdere foto's zichtbaar te maken, eventueel met landmarks en gekleurde gebieden. Een dergelijke compositie is op te slaan om later te tonen.

Het is verder mogelijk om papers aan projecten te koppelen en het systeem is voorzien van hulpfunctisch zoals een zoomscherm en het ongedaan maken van acties. 

Bij met maken van het systeem is gebruik gemaakt van een Django webservice die in Python werkt en webpagina's genereerd die op de client een JavaScript-applicatie draait. Voor \'e\'en functie is een Flash-applicatie geschreven. De onderliggende database is een MySQL-database.

%Anders nog iets? -Bastiaan
