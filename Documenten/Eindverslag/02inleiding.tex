\section{Inleiding}
\label{inleiding}
De afgelopen tien weken hebben wij met elkaar gewerkt aan een webapplicatie voor het \casamproject. Voor het vervullen van deze opdracht hebben wij op diverse terreinen verschillende uitdagingen moeten overwinnen. Dit ging van de communicatie met experts van het Erasmus Medisch Centrum die geen verstand hadden van de IT-wereld, tot aan het werken met programmeertalen waar niemand ervaring mee had. 
Uiteindelijk hebben we een product gebouwd dat vele verschillende programmeertalen bij elkaar brengt en op deze manier gebruiken we de verschillende krachten in een nieuwe combinatie. Ook hebben wij het idee dat het product voldoet aan de eisen van het \casamproject. Dit resultaat hebben wij ook te danken aan het Agile programmeerprincipe waardoor wij veel tussentijds contact hebben gehad met de opdrachtgever en de uiteindelijke gebruikers.
\\
\\
In dit verslag besteden wij aandacht aan de verschillende uitdagingen die voor ons lagen, de oplossingen die we daarvoor hebben gevonden en het research wat we daarvoor hebben moeten doen. In sectie 2 leggen we uit wat de probleemstelling is en daaraan gekoppeld wordt in sectie 3 een analyse gemaakt van het probleem en een oplossingsrichting gedefinieerd. In sectie 4 zullen we ingaan op de planning die we hebben gemaakt, om in sectie 5 in te gaan op de aanpak van ons probleem. Daarna zullen we in sectie 6 een toelichting geven op onze implementatie en een aantal bijzondere aspecten uitlichten. Om daarna in sectie 7 een analyse te maken over de mogelijkheden om het product uit te breiden en te verbeteren. 
\\
\\
Bij dit verslag horen enkele document welke zijn toegevoegd als bijlage. Voorbeelden hiervan zijn het Plan van Aanpak, het orientatieverslag en het RAD-document van de database applicatie. In het verslag zal op verschillende momenten worden verwezen naar deze documenten. 
