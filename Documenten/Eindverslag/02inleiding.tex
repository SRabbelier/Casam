\section{Inleiding}
\label{inleiding}
De afgelopen tien weken hebben wij met elkaar gewerkt aan een webapplicatie voor het \casamproject. 
Voor het uitvoeren van deze opdracht hebben wij op diverse terreinen verschillende uitdagingen moeten overwinnen. 
De uitdaging lag in de communicatie met medische experts van het Erasmus Medisch Centrum die geen verstand hebben van de IT-wereld tot aan het werken met programmeertalen waar niemand ervaring mee had. 
Uiteindelijk hebben we een product gebouwd dat veel verschillende programmeertalen bij elkaar brengt en op deze manier gebruiken we de verschillende krachten in een nieuwe combinatie. 
Wij zijn er van overtuigd dat het product voldoet aan de eisen van het \casamproject. 
Dit resultaat hebben wij te danken aan het Agile programmeerprincipe waardoor wij veel tussentijds contact kunnen hebben gehad met de opdrachtgever en de uiteindelijke gebruikers.
\\
\\
In dit verslag besteden wij aandacht aan de verschillende uitdagingen die voor ons lagen, de oplossingen die we daarvoor hebben gevonden en het research dat we daarvoor hebben moeten doen. 
In sectie 2 leggen we uit wat de probleemstelling is en daaraan gekoppeld wordt in sectie 3 een analyse gemaakt van het probleem en een oplossingsrichting gedefinieerd. 
In sectie 4 zullen we ingaan op de planning die we hebben gemaakt, om in sectie 5 in te gaan op de aanpak van het probleem. 
Daarna zullen we in sectie 6 een toelichting geven op onze implementatie en er een aantal bijzondere aspecten uitlichten. 
In sectie 7 maken we een analyse te maken van de mogelijkheden om het product uit te breiden en te verbeteren. 
\\
\\
Bij dit verslag horen enkele documenten die zijn toegevoegd als bijlagen. 
Voorbeelden hiervan zijn het Plan van Aanpak, het orientatieverslag en het RAD-document van de database applicatie. 
In het verslag zal op verschillende momenten worden verwezen naar deze documenten. 
