\section{Implementatie}
\label{Implementatie}
In de tweede week van het project zijn we echt begonnen met het bouwen van het systeem. We zijn begonnen met zogenaamde spike solutions (snelle simpele oplossingen om een deelprobleem op te lossen) van verschillende basis functionaliteiten, zoals het uploaden van images en het plaatsen van landmarks geplaatste images met behulp van drag \& drop. In een later stadium hebben we voor de grotere uitdagingen spike solutions geschreven, hieronder vallen bijvoorbeeld het morphen van plaatjes en het visualiseren van het Point Distribution Model. Als een spiked solutions werkte werd deze aan het bestaande systeem gekoppeld, op deze manier hadden we in een vrij korte tijd een basis systeem met haar basis functionaliteiten. Zo konden we ons vervolgens concentreren op de Javascript functies welke zouden gaan zorgen voor de AJAX afhandeling en de visuele effecten.

\subsection{Project gerelateerd}
Daar waar het mogelijk en handig was hebben we geprogrammeerd in tweetallen. Tijdens de implementatie fase hebben we veel gebruik gemaakt van de beschikbare whiteboards, deze waren handig voor het brainstormen en als iemand vast liep dan kon het probleem getekend en uitgelegd worden. Dit alles zorgde voor een goede synergie en hield iedereen de motivatie die hij had goed vast.
Ook tijdens de implementatie fase hebben we veel contact gehad met de leden van het \casamproject en zijn er verschillende tussentijdse presentaties geweest. Steeds hebben we na deze presentaties een uitgebreid overleg gehad over de stand van zaken en ge\"{e}valueerd over hoe het project er voor stond. In sommige gevallen hebben we ook onze prioriteitenlijst aangepast aan de hand van deze vergaderingen.

\subsection{Database systeem}
Lorem ipsum

Voorbeeld van de implementatie van de bitmap creator python-javascript-html-flash-javascript-python

Op het eind van het project hadden we veel bereikt. Maar ook is er werk blijven liggen. Voorbeelden hiervan zijn: de ViewMode voor artsen en de analyse van statische data.

\subsection{Image Processing}
Dit was een van de grootste uitdagingen van het project. Niemand had ervaring met de verschillende bibliotheken of uitgebreide ervaring met dit aspect van Image Processing. Er waren verschillende opties om de benodigde technieken te implementeren, zo konden we ze zelf implementeren met behulp van Numpy\cite{numpy}, of gebruik maken van grotere libraries zoals Insight Segmentation and Registration Toolkit (ITK)\cite{ITK} of de Visualization Toolkit (VTK)\cite{vtk}. Verder was er nog de Python Imaging Library (PIL)\cite{pil}, die geschikt is om eenvoudige beeldbewerking te doen. Uiteindelijk is er, mede op advies van Dr. Botha, gekozen voor PIL in combinatie met de VTK library. Om aan de vraagstelling te kunnen voldoen (het in kaart brengen van de variaties per landmark en het kunnen projecteren van getekende gebieden op ��n gemiddeld plaatje) hebben we gekozen om een deel van de methode Active Shape Modeling, genaamd Point Distribution Models, toe te passen in combinatie met Thin Plate Spline transformaties. Hierdoor is het mogelijk voor de onderzoeker om op een wiskundige verantwoorde manier anatomische structuren op een gemiddelde te projecteren en de variaties per landmark te visualiseren.
Al snel bleek dat er veel research nodig was om de theorie achter Principal Component Analysis, Active Shape Models, Point Distribution Models en de Thin Plate Spline Transformaties te doorgronden. Ook kostte het enige extra tijd om met de VTK bibliotheek zelf bekend te raken.

\subsubsection{Point Distribution Model}
Om het Point Distribution Model te maken is een set nodig van verschillende landmarks waarvan de variaties en de gemiddelden interessant zijn om weer te geven. Het is hierbij van belang om de landmarks zo te kiezen dat ze makkelijk zijn te vinden in elke foto en belangrijke punten aangeven op de structuur waar het om gaat. De gebruiker kan landmarks in de foto aangeven door eerst een landmarktype aan te maken, dan een landmark van een bepaald type en hierbij aan te geven of het een shapedefining landmark is of niet. Nadat de gebruiker in de foto's de gewenste landmarks heeft aangegeven kan hij door 'Analyse selected landmarks' aan te klikken het Point Distribution Model van de geselecteerde landmarks uitrekenen en weergeven.

Op dit moment wordt in JavaScript gecontroleerd welke images en landmarks geselecteerd zijn en wordt via een AJaX request de id's van de image en de geselecteerde landmarks per image doorgegeven. Aan de hand van de id's van de image en de landmarks worden vervolgens de bijbehorende objecten uit de database gehaald. In de view wordt getest of de geselecteerde landmarks wel geschikt zijn om een Point Distribution Model van te maken. Dit is bijvoorbeeld niet zo als er geen landmarks geselecteerd zijn, als er te weinig landmarks geselecteerd zijn van een type of als de geselecteerde landmarks van een geheel ander type zijn. In elk van deze gevallen wordt er een foutmelding aan de gebruiker geretourneerd in de vorm van een JavaScript alert met de aard van de fout, zodat de gebruiker zijn selectie kan aanpassen.

Als echter de juiste landmarks zijn geselecteerd worden allereerst via de co\"{o}rdinaten van de landmarks per image een vtkUnstructuredGrid aan gemaakt met de desbetreffende punten erin als vertices. Vervolgens wordt over deze grids de Generalized Procrustes Analysis uitgevoerd via het vtkProcrustesAlignmentFilter. Dit zorgt ervoor dat de gevonden modi van variatie onafhankelijk van de positie en de rotatie van de objecten zijn. Door ervoor te kiezen om de mode op RigidBody te zetten blijft echter wel de grootte van de structuren behouden. De opgeslagen grids zijn nu iteratief getransleerd en geroteerd naar elkaar toe.

Nu is het tijd om de Principal Component Analysis (PCA) op de aligned grids uit te voeren, dit is gedaan met behulp van het vtkPCAAnalysisFilter. Uit het resultaat kunnen de gemiddelde posities per landmark gehaald worden. We hebben er voor gekozen om de eerste twee modi van variatie te berekenen, de hoofdmodus van variatie (eerste mode) zorgt altijd voor de grootste verandering en is de eigenvector die uit de PCA komt met de grootste eigenwaarde. Om deze variaties te berekenen vragen we de GetParameterisedShape en berekenen we hiermee voor de eerste en tweede modus de extremen, dat wil zeggen plus en min 3 standaarddeviaties van het gemiddelde.

Nu we de benodigde co\"{o}rdinaten hebben (gemiddelde en extremen) kunnen we beginnen aan de visualisatie van de landmarks. Dit wordt gedaan met behulp van de PIL. Aan de hand van de afmetingen van de originele images en de gegeven co\"{o}rdinaten word voor het gemiddelde een ellipse getekend. Tussen de extremen van de twee hoofdmodi van variatie wordt een lijn getekend. De lengte van deze lijn geeft de grootte van de mogelijke variatie aan op basis van de voorbeelden uit de gegeven landmarks. Door de ellipsoids en lijnen weer te geven met een doorzichtige achtergrond is het vervolgens mogelijk om deze als een overlay over de images te projecteren. De overlay word opgeslagen in de database is beschikbaar voor weergave vanuit de interface.


\subsubsection{Thin Plate Spline Transformatie}


\subsubsection{Tekenen}



