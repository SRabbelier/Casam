\section{Implementatie}
\label{Implementatie}
In de tweede week van het project zijn we echt begonnen met het bouwen van het systeem. We zijn begonnen met spiked solutions van verschillende basis functionaliteiten, zoals het uploaden van images en het plaatsen van landmarks geplaatste images met behulp van drag \& drop. In een later stadium hebben we voor de grotere uitdagingen spiked solutions geschreven, hieronder vallen bijvoorbeeld de Image morphing. Als een spiked solutions werkte werd deze aan het bestaande systeem gekoppeld, op deze manier hadden we in een vrij korte tijd een basis systeem met haar basis functionaliteiten. En konden we ons concentreren op de Javascript functies welke zouden gaan zorgen voor de AJAX afhandeling en de visuele effecten. 

\subsection{Project gerelateerd}
Daar waar het mogelijk en handig was hebben we geprogrammeerd in tweetallen. Tijdens de implementatie fase hebben we veel gebruik gemaakt van de beschikbare whiteboards, deze waren handig voor het brainstormen en als iemand vast liep dan kon het probleem getekend en uitgelegd worden. Dit alles zorgde voor een goede synergie en hield iedereen de motivatie die hij had goed vast.
Ook tijdens de implementatie fase hebben we veel contact gehad met de leden van het \casamproject en zijn er verschillende tussentijdse presentaties geweest. Steeds hebben we na deze presentaties een uitgebreid overleg gehad over de gang van zaken en ge�valueerd over hoe het project er voor stond. In sommige gevallen hebben we ook onze prioriteitenlijst aangepast aan de hand van deze vergaderingen. 
\subsection{Database systeem}
Lorem ipsum

Voorbeeld van de implementatie van de bitmap creator python-javascript-html-flash-javascript-python

Op het eind van het project hadden we veel bereikt. Maar ook is er werk blijven liggen. Voorbeelden hiervan zijn: de ViewMode voor artsen en de analyse van statische data.
\subsection{Image Processing}
Na heel veel research en veel theoretische papers konden we hier gaan proberen om VTK werkend te krijgen en een aantal spiked solutions te schrijven hiervoor. Door een gebrek aan goede documentatie en heldere uitleg heeft dit alles veel meer tijd gekost dan we hadden verwacht. 

Uiteindelijk zijn een aantal onderdelen van de Image Processing blijven liggen. Voorbeelden hiervan zijn de transformatie naar de Standard Anatomical View.

