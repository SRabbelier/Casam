\section{Implementatie}
\label{Implementatie}
In de tweede week van het project zijn we echt begonnen met het bouwen van het systeem. We zijn begonnen met spiked solutions van verschillende functionaliteiten. Later hebben we de verschilende spikes in \'{e}\'{e}n systeem bij elkaar gebracht. In vrij korte tijd lag de Django basis voor de applicatie er al en zijn we voornamelijk bezig geweest met het schrijven aan de Javascript zodat de interface zo net mogelijk zou worden. Ook is er in die fase gebouwd aan de flash applicatie die bitmaps zou gaan tekenen.
\subsection{Project gerelateerd}
Daar waar het mogelijk en handig was hebben we geprogrammeerd in tweetallen. Tijdens de implementatie fase hebben we veel gebruik gemaakt van de beschikbare whiteboards, deze waren handig voor het brainstormen en als iemand vast liep dan kon het probleem getekend en uitgelegd worden. Dit alles zorgde voor een goede synergie en hield iedereen de motivatie die hij had goed vast.
Ook tijdens de implementatie fase hebben we veel contact gehad met de leden van het CASAM project en zijn er verschillende tussentijdse presentaties geweest. Steeds hebben we na deze presentaties een uitgebreid overleg gehad over de gang van zaken en ge�valueerd over hoe het project er voor stond. In sommige gevallen hebben we ook onze prioriteitenlijst aangepast aan de hand van deze vergaderingen. 
\subsection{Database systeem}
Lorem ipsum

Voorbeeld van de implementatie van de bitmap creator python-javascript-html-flash-javascript-python

Op het eind van het project hadden we veel bereikt. Maar ook is er werk blijven liggen. Voorbeelden hiervan zijn: de ViewMode voor artsen en de analyse van statische data.
\subsection{Image Processing}
Na heel veel research en veel theoretische papers konden we hier gaan proberen om VTK werkend te krijgen en een aantal spiked solutions te schrijven hiervoor. Door een gebrek aan goede documentatie en heldere uitleg heeft dit alles veel meer tijd gekost dan we hadden verwacht. 

Uiteindelijk zijn een aantal onderdelen van de Image Processing blijven liggen. Voorbeelden hiervan zijn de transformatie naar de Standard Anatomical View.

