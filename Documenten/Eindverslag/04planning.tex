\section{Planning}
\label{Planning}

We begonnen het project met het maken van een planning voor de komende tien
weken zodat er voor zowel de opdrachtegever als onszelf duidelijk was wanneer
er wat gedaan zou worden. In verband met de Agile methoden zijn de eerste paar
weken in meer detail gepland dan de latere weken met de bedoeling om de
planning steeds bij te werken voor de komende (paar) weken. Deze voorgenomen
planning wordt besproken in paragraaf \ref{voorgenomen_planning}. Zoals
verwacht is deze ruweplanning aangepast in de weken daarna, deze realisatie
wordt besproken in \ref{werkelijke_planning}.

\subsection{Voorgenomen planning}
\label{voorgenomen_planning}

Doordat we in het begin slechts een grove planning hebben gemaakt waren vooral
de eerste paar weken in detail uitgewerkt. Er moesten twee documenten af zijn
voordat er met de implementatie van het product kon worden begonnen, dit zijn
het plan van aanpak (PvA) en het requirement analysis document (RAD). Tevens
hebben we ons direct voorgenomen een demo in de eerste weken van het project te
maken zodat de opdrachtgever snel feedback kon geven op de richting waar het
project in gaat.
Aangezien de tentamens op 16/06 beginnen is afgesproken
voor deze datum het project af te ronden
\subsubsection{Plan van Aanpak}

Tijdens de eerste meeting is het volgende schema voor het PvA afgesproken,
waarbij de deadlines zijn afgesproken met de gelegenheid om verbeteringen
door te voeren in het achterhoofd.

09/04: Concept PvA mailen
14/04: Verwerken feedback PvA
15/04: Meeting in erasmus
17/04: PvA af

\subsubsection{Requirement Analysis Document}

Tevens is voor het RAD een (minder granulaire) planning gemaakt, deze overlapt
met de planning voor het PvA gezien het schrijven van het tweede document en
het verwerken van verbeteringen aan een ander document samen kunnen gaan.

Nadat de tweede ronde van feedback op het PvA was verwerkt werd begonnen met
het RAD, zodat deze zo snel mogelijk af is. Eventuele feedback op het PvA
hierna wordt eerst verwerkt.

27/04: RAD af

\subsubsection{Database Applicatie}

Om een demo te kunnen geven waarin zo veel mogelijk geevalueerd kan worden was
besloten om de twee weken na het inleveren van het RAD te besteden aan
het maken van spike solutions en deze te integreren in een simpele applicatie
die deze functionaliteit presenteerd.

08/05

\subsubsection{Active Shape Modelling}

Een belangrijk deel van de applicatie is het weergeven van de hoofdmodi van
variatie purr landmark type en het weergeven van de gemiddelde landmarks.

25/05: Active Shape Modelling af.

\subsubsection{Image Processing}

Tevens is het morphen en tekenen van gebieden cruciaal voor de aangeboden
oplossing.

11/06: Image Processing af.

\subsubsection{Eindverslag}

Het schrijven van het eindverslag moet ruim voor de eindpresentatie afgerond zijn
zodat het verslag goedgekeurd kan worden voor aanvang van de eindpresentatie.

12/06 Eindverslag af.

\subsubsection{Eindpresentatie}

De eindpresentatie is de uiteindelijke afronding van het project. 

19/06: Eindpresentatie gegeven.

\subsection{Werkelijke planning}
\label{werkelijke_planning}

In deze sectie geven wij aan hoe de opgestelde planning aansluit op de
uiteindelijke realisatie.

Zowel het PvA als het RAD zijn op tijd af gekomen, de andere milestones zijn
wat gewijzigd. Aangezien verschillende leden van de projectgroep tentamens
hebben in de weken vanaf 15 juni is besloten om het project rond deze datum af
te ronden. Hierdoor schoven de deadlines voor het schrijven van het eindverslag
en het geven van de eindpresentatie op naar voren. In de originele planning was
geen rekening gehouden met het systeem ook te laten draaien op een server in
het Erasmus MC. Hierdoor moet dit parallel plaatsvinden tijdens het schrijven
van het verslag. Door de onverwachte moeilijkheid van de image processing
libraries liep de implementatie van het programma sterk uit.

