\section{Planning}
\label{Planning}

We begonnen het project met het maken van een planning voor de komende 10 weken
zodat er voor zowel de opdrachtegever als onszelf duidelijk was wanneer er wat
gedaan zou zijn. Ivm met de Agile methoden zijn de eerste paar weken in meer
detail geplanne dan de latere weken met de bedoeling om de planning steeds bij
te werken voor de komende (paar) weken. Deze 'voorgenomen' planning wordt
besproken in paragraaf \ref{voorgenomen_planning}. Zoals verwacht is deze ruwe
planning aangepast in de weken daarna, deze planning wordt besproken in
\ref{werkelijke_planning}.

\subsection{Voorgenomen planning}
\label{voorgenomen_planning}

Doordat we in het begin slechts een grove planning hebben gemaakt waren vooral
de eerste paar weken in detail uitgewerkt. Er moesten twee 'documenten' af zijn
voordat er met de implementatie van het product kon worden begonnen, het 'plan
van aanpak' (PvA) en het 'requirement analysis document' (RAD). Tevens hebben
we ons direct voorgenomen een 'demo' in de eerste weken van het project af te
hebben zodat de opdrachtgever snel feedback kan geven op de richting waar het
project in gaat.

\paragraph{Plan van Aanpak}

Tijdens de eerste meeting is het volgende schema voor het PvA afgesproken,
waarbij de deadlines zijn afgesproken met de gelegenheid om verbeteringen
door te voeren in het achterhoofd.

09/04: Concept PvA mailen
14/04: Verwerken feedback PvA
15/04: Meeting in erasmus
17/04: PvA af

\paragraph{Requirement Analysis Document}

Tevens is voor het RAD een (minder granulaire) plannin gemaakt, deze overlapt
met de planning voor het PvA aangezien het schrijven van een nieuw document en
het verwerken van verbeteringen aan een ander document samen kunnen gaan.

Nadat de tweede ronde van feedback op het PvA is verwerkt zal worden begonnen
met het RAD, zodat deze zo snel mogelijk af is. Eventuele feedback op het PvA
hierna wordt eerst verwerkt (werk aan het RAD moet dan wachten).

27/04: RAD af

\paragraph{Demo}

Om een degelijke demo te kunnen geven waarin zo veel mogelijk geevalueerd kan
worden was besloten om twee weken na het inleveren van het RAD te besteden aan
het maken van 'spike solutions' en deze te integreren in een simpele applicatie
die deze functionaliteit showcased.

\subsection{Werkelijke planning}
\label{werkelijke_planning}

In deze sectie geven wij aan welke planning wij hadden en hoe deze aansluit op
de uiteindelijke realisatie.
