\section{Bijlagen}
\label{Bijlagen}
\subsection{Prioriteiten}
\begin{multicols}{2}
\Large{\textbf{Database systeem en basic invoer}}
\large{\textbf{MOET}}
\begin{enumeration}
	\item Uploaden en verwijderen van foto's
	\item Invoeren en opslaan van landmarks (bijvoorbeeld met behulp van drag en drop)
	\item Managen van projecten en selectie door middel van tags
	\item Managen van MogelijkeMetingen en deze koppelen aan projecten
	\item Invoeren en opslaan van statistische informatie
	\item User control (login)
	\item User management, met een vooraf ingesteld auto-logoff en OK-mode
\end{enumeration}

\Large(\textbf{Image Processing}}
\large{\textbf{MOET}}
\begin{enumeration}
	\item Automatisch schalen, draaien en verplaatsen
	\item Morphen naar standaard AnatomicalView
	\item Visualisering van landmarks, lines en geometrische vormen
	\item Samenstellen te tonen van layering
	\item Analyse van statistische informatie (bijvoorbeeld van landmarks)
\end{enumeration}
\large(\textbf{EVENTUEEL}}
\begin{enumeration}
	\item Aligning van (X-ray en) handmatige tekeningen over Standard Anatomical View
	\item Achtergrondverwijdering
	\item Visueel tonen van statistische analyse (grafieken, tabellen, visualisering op lichaam)
	\item Proprocessing (witbalans)
\end{enumeration}
\large(\textbf{BUITEN SCOPE}}
\begin{enumeration}
	\item Detecteren liniaal en uitrekenen schaling
	\item 3D-draaiing correctie, camerahoekcorrectie
	\item Automatische detectie gekleurde landmarks
\end{enumeration}
\end{multicols}

\subsection{Ori\"{e}ntatieverslag}

\subsection{Plan van Aanpak}

\subsection{Requirements Analysis Document}
