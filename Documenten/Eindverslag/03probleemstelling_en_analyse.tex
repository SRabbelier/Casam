\section{Probleemstelling en Analyse}
\label{Probleemstelling_en_analyse}
\subsection{Probleemstelling}
Het \casamproject was op zoek naar een applicatie die hun kon helpen bij de anatomische vraagstukken die zij hadden. 
Het ging hierbij vooral om hulp bij het warpen en morphen van foto's en de mogelijkheden om op verschillende foto's bijzonderheden aan te kunnen geven en deze met elkaar te vergelijken. 
Al in het eerste gesprek bleek dat er op dat moment nog geen enkele applicatie lag en dat alle foto's gewoon werden opgeslagen op een harde schijf en dat er out-of-the-box software werd gebruikt voor de foto manipulatie. 
Op dat moment hebben wij voorgesteld om naast de foto manipulatie ook aandacht te besteden aan de administratie en opslag van het materiaal.
Tenslotte kwamen zij ook met de vraag of het mogelijk was om bepaald materiaal dat in het ziekenhuis werd gemaakt, zoals r\"{o}ntgenfoto's, direct te importeren in het systeem. Ook wilde zij graag publicaties die hoorde bij een project erbij willen opslaan.

\subsection{Analyse}
In overleg met het \casamproject hebben we toen afgesproken om het project in drie delen op te delen: een database applicatie, Image processing en tenslotte de koppeling met andere systemen. 
Daarnaast werd meteen afgesproken dat de project groep op de TU Delft zou werken, dit omdat de facilitaire mogelijkheden in Delft beter zijn dan op het EMC. 
Om wel voldoende feedback te blijven ontvangen is er gekozen voor een Agile programmeerprincipe, op deze manier zou de opdrachtgever voldoende betrokken blijven bij het proces.

In de eerste week van het project hebben wij een lijst opgesteld van de verschillende functionaliteiten die in het systeem moesten komen. 
Van te voren was al duidelijk dat niet alle functionaliteiten gebouwd konden worden binnen de scope van dit project. 
Daarom zijn in samenspraak met de opdrachtgever de functionaliteiten van prioriteiten voorzien, uiteindelijk is daar een diagram uitgekomen, zie hiervoor bijlage A.

Mede op basis van het prioriteiten diagram hebben wij de verschillende uitdagingen gedefinieerd:
\begin{itemize}
  \item Project gerelateerd
  \begin{itemize}
    \item Gebrek aan medische en anatomie kennis
    \item Gebrek van IT kennis aan de kant van de opdrachtgever
    \item Samenwerking in een grote groep
  \end{itemize}
  \item Database systeem
  \begin{itemize}
    \item Gebruikersvriendelijkheid
    \item Van diverse locaties bereikbaar
    \item Eenvoudige manier om data toe te voegen aan het systeem
    \item Schaalbaarheid
  \end{itemize}
  \item Image processing
  \begin{itemize}
    \item Variaties in landmarks
    \item Morphen van verschillende foto's naar \'{e}\'{e}n gemiddelde foto
    \item Verplaatsen en roteren van foto's naar standard AnatomicalView
  \end{itemize}
\end{itemize}
In de volgende secties lichten wij toe hoe we met deze uitdagingen zijn omgegaan.
