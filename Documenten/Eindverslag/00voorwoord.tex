%\addcontentsline{toc}{section}{Voorwoord}

\section{C.A.S.A.M. werking door samenwerking.}
In 2008 begon Anton Kerver met zijn afstudeeropdracht: het in kaart brengen van een gebied waar veilig ('safe zone') een laserbehandeling voor spataderen kan plaatsvinden.
\\
\\
Al snel kwam de vraag: hoe kunnen wij de enorme hoeveelheid data bruikbaar (lees: visueel) maken  voor de gebruiker, de (in dit geval) vaatchirurg. Toen Anton op het Internet ging surfen om een manier te vinden van datavisualisatie kwam hij terecht bij commercieel beschikbare 'Morphing' programma's. Hij wist, als (op de medische faculteit) erkende whizzkid snel de principes te doorgronden van het programma en ging vrolijk aan de slag... prachtige platen en echt prachig in beeld gebrachte safe zones waren het gevolg... alle problemen opgelost... klaar is Kees.
\\
\\
...totdat de vragen gecompliceerder werden... HOE worden de beelden gemorpht, welke mate van precisie wordt gehanteerd, hoe krijgen we het systeem beschikbaar, hoe kan de individuele chirurg (en dus de pati\"{e}nt) tijdens een operatie profiteren van deze data�enfin Kees was helemaal niet klaar...
Historisch gezien ging het bij ons op de afd. Neuroscience-Anatomie altijd zo: onoplosbaar technisch probleem? Delft bellen... en wel onmiddellijk! Instant succes volgt dan weliswaar niet maar meestal binnen een maandje of twee is er het verlossende woord uit Delft.
\\
\\
Zo ook nu dus. Om een lang verhaal kort te maken... na een zoektocht door de verschillende faculteiten en afdelingen van de TU Delft kwamen we terecht bij Charl Botha en Frits Post v.d. fac. Electrical Engineering, Mathematics and Computerscience,  afd. Medical Visualisation. Precies raak: enthousiaste mensen die ons gelijk op ons gemak stelden... neen, wij hadden geen onwezenlijke wensen en ja, zij konden ons zeker helpen. Charl, heeft na een collegecyclus over het onderwerp medische data-visualisatie een oproep gedaan onder zijn studenten om een bachelorproject te doen met onze vraagstelling als uitgangspunt.
\\
\\
Binnen 3 weken was daar onze groep studenten...  6 enthousiastelingen die stonden te popelen om ons te helpen... de IT termen vlogen over de tafel en na een kwartier werd het duidelijk... wij medici begrepen totaal niet waar ze het over hadden en zij wisten in het geheel niet wat wij wilden, maar...... dit wordt mooi! En mooi werd het. Bastiaan Bijl, Ben Sedee, Jaap den Hollander, Noeska Smit, Sjors van Berkel en Sverre Rabbelier gingen aan de slag, hadden veel contact met Anton en tussentijds vielen onze monden al open, maar wat er na een, toch beperkte tijd, van 10 weken nu voor ons ligt overtreft onze stoutste verwachtingen. 
\\
\\
Wat vooral opviel was de unieke samenkomst van talenten: ieder lid van de groep had zijn eigen specialisme (van communicatie met de leek tot wiskundige expertise tot effectief project management) en niet alleen brachten zij hun eigen kennis in�de anderen hadden de positief-actieve instelling om zich deze kennis eigen te maken. Een echte zelflerende groep met respect voor elkaars identiteit. Met als resultaat: een fantastische oplossing voor ons probleem en een geweldige basis voor uitbreiding naar een web-based consultatiesysteem voor chirurgen met direct positieve effecten voor de (chirurgische)pati\"{e}nt! Met dit programma kunnen we jarenlang vooruit�alle chirurgische toegangen tot het menselijk lichaam kunnen nu op een gefundeerde wijze, compleet met statistische onderbouwing worden bestudeerd en ter beschikking worden gesteld, waarmee de Computer Assisted Surgical Anatomy Mapping (C.A.S.A.M.) droom werkelijkheid kan worden. Bastiaan, Ben, Jaap, Noeska, Sjors en Sverre heel veel dank voor jullie enthousiasme en inzet en voor de ervaring hoe een groep gemotiveerde specialisten tot grote hoogte kunnen komen door onzelfzuchtige, echte samenwerking. 
\\
\\
Onze Hartelijke dank!
\\
\\
Dr. Gert-Jan Kleinrensink
\\
Drs. Anton Kerver
