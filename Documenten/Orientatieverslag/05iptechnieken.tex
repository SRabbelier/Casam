\section{Image Processing Technieken}
\label{Image Processing Technieken}
In dit hoofdstuk gaan we in op de theorie achter de gebruikte Image Processing technieken. 
Allereerst beginnen we met een beschrijving van het Point Distribution Model, dit doen we aan de hand van Generalised Procrustes Analysis en Principal Component Analysis. 
Tot slot bespreken we kort de theorie achter de Thin Plate Spline transformatie. 

\subsection{Point Distribution Model}
Het Point Distribution Model is een model dat de gemiddelde vorm en modes van variatie kan weergeven.\cite{PDM} 
Een vorm is hierbij een vorm alles wat overblijft als alle locatie, schaling en rotatie effecten eruit gefilterd zijn.\cite{gpa}
Om een vorm te beschrijven aan de hand van een set van voorbeelden is er een manier nodig om de vorm te defini\"{e}ren. 
Dit is mogelijk door de expert met landmarks belangrijke punten in de vorm aan te geven. 
Een landmark is in dit geval een punt dat duidelijk herkenbaar moet zijn in elke foto. 
Dit kan een bony landmark (komt voort uit de anatomie van de mens) of een zelf gekozen landmark. 
Een voorbeeld van een landmark bij een foto van de hand kan het topje van de duim zijn. 
Gegeven een set van \textit{n} landmarks aangegeven in een aantal voorbeelden van een vorm kan er een statistisch shape model worden opgesteld. 
Om dit te bereiken moeten de foto's eerst in eenzelfde co\"{o}rdinatenframe worden gezet. 
Dit wordt gedaan met Generalized Procrustes Analysis.

\subsubsection{Generalised Procrustes Analysis}
Procrustus Analysis kan worden gebruikt om 2 sets van landmarks uit te lijnen. 
Generalised Proctrustes Analysis echter kan worden gebruikt om \textit{k} sets van landmarks met elkaar uit te lijnen. 
Het transleert, roteert en schaalt (eventueel) elke vorm door de som van de gekwadrateerde afstanden tot het gemiddelde te minimaliseren. 

Het algoritme van Generalised Procrustus Analysis werkt als volgt\cite{gpa}:
\begin{enumerate}
	\item Selecteer 1 vorm als tijdelijk gemiddelde vorm
	\item Lijn de overige shapes uit met deze vorm:
	\begin{itemize}
		\item Bereken het centroid van elke vorm.
		\item Lijn alle vormen uit naar de oorsprong.
		\item Normaliseer de grootte van de centroid van elke vorm.
		\item Roteer de vorm om uit te lijnen met het nieuwste geschatte gemiddelde.
	\end{itemize}
	\item Bereken het nieuwe geschatte gemiddelde uit de uitgelijnde vormen.
	\item Als het geschatte gemiddelde uit stap 2 en 3 anders zijn, ga dan naar 2 terug. Als dit niet zo is is het ware gemiddelde gevonden.
\end{enumerate}

Elke vorm kan nu gerepresenteerd worden als een \textit{2n} element vector:
$x = (x_{1},...,x_{n},y_{1},...,y_{n})$

De uitgelijnde trainingsets vormen nu een wolk in de \textit{2n} dimensionale ruimte. 
Door Principal Component Analysis uit te voeren kunnen de hoofdassen van de wolk worden berekend.

\subsubsection{Principal Component Analysis}

Principal Component Analysis (PCA) is een manier om patronen in data te herkennen, en de overeenkomsten en verschillen binnen de dataset weer te geven.\cite{pca}
Als je van alle vectoren de covariantiematrix \textit{S} uitrekent en hierop eigen-analyse uitvoert wordt de vorm van de \textit{m x n}-dimensionale puntenwolk een gewogen som van orthogonale vectoren. 
Hierin zijn de gewichten de eigenwaarden $\lambda$ en de 'assen' die de ruimte opspannen de eigenvectoren \textit{p}. 
De eigenvectoren waarbij de grootste eigenwaarden horen staan hierbij voor de grootste variaties in de dataset. 
De eigenvector met de hoogste eigenwaarde is de Principal Component, de meest significante relatie tussen de dimensies van de data. 
De overige eigenvectoren p representeren de andere principal directions. Elk punt op het oppervlakte van het object kan nu gerepresenteerd worden als een lineaire combinatie van p toegevoegd aan de centroid of het gemiddelde van de vorm.

Het PCA algoritme:
\begin{itemize}
\item Verkrijg de datapunten
\item Trek er het gemiddelde van af
\item Bereken de covariantie matrix
\item Bereken de eigenvectoren en waarden
\end{itemize}

Nu kunnen we met de belangrijkste modi van variatie (de eigenwaarden met de bijbehorende eigenvectoren) het shape model defini\"{e}ren:

$x = x_{mean} + Pb$

Hierbij is $x_{mean}$ het gemiddelde van de uitgelijnde shape parameteres, \textit{P} is een \textit{2n x t} matrix waarvan de kolommen de t unit vectoren van de principal assen van de wolk vormen. \textit{b} is een t-dimensionale vector gegeven door:

%$b = P^{T}(x-\={x})$

Nu is het mogelijk om nieuwe shape voorbeelden te constru\"{e}ren door de shape parameters te vari\"{e}ren. Als er grenzen gesteld worden aan het vari\"{e}ren van de parameters b blijft de gegenereerde shape in de verwachting van de mogelijke shapes. De grenzen worden hierbij gesteld op plus en min drie standaarddeviaties van het gemiddelde.


\subsection{Thin Plate Spline Transform}
Om de tekeningen daadwerkelijk te gaan morphen gebruiken we een Thin Plate Spline transformatie. De naam Thin Plate Spline 
komt van een vergelijking met het buigen van een dunne metalen plaat.\cite{tsp} 