\section{Software}
\label{Software}
In dit hoofdstuk bespreken we de softwaretechnologi/"{e}n die we gaan gebruiken om de applicatie te maken. 
We beginnen met de basis van de applicatie door eerst de IDE, Python en Django te bespreken. 
Hierna bespreken we de 2 gebruikte bibliotheken, VTK en PIL. 
Tot slot bespreken we kort JavaScript en Flash.

\subsection{Python}
\label{software_python}
Het programma is geschreven in python, een dynamische programeer taal met goede ondersteuning voor 3rd party libraries in python zelf, maar ook in C of C++. De taal is zo opgebouwd dat het makkelijk is om prototypes te maken en die later door middel van refactoring om te schrijven naar een compleet eindproduct. Door de ondersteuning van onder andere objectgeorienteerd programmeren maar ook functioneel programmeren is het voor vele programmeurs een makkelijke taal om te leren.

\subsection{Django}
Met behulp van Django wordt python een taal waarmee eenvoudig web applicaties geschreven kunnen worden zonder dat de python code zelf 'vervuild' wordt met html/css/etc. De applicatie maakt gebruik van html templates waarin met een specifieke syntax data van de python kant ingevoegd kan worden. Tevens kan door middel van overerving een consistente web applicatie geschreven worden zonder dat er (html) code gekopieerd en geplakt hoeft te worden. Django heeft ook een uitgebreide forms module waarmee html formulieren gemaakt en verwerkt kunnen worden. Onderdeel van dit verwerken is ook het valideren van de gegevens die door de gebruiker aangeleverd worden. Tenslotte heeft Django ondersteuning voor het escapen van html zodat door de gebruiker ingevoerde text nooit kan resulteren in bijvoorbeeld cross-site scripting (XSS).

\subsection{Integrated Development Environment (IDE): Eclipse}
Eclipse, in samenwerking met de 'pydev' plugin, is op het moment een van de beste python editors. Het heeft niet alleen ondersteuning voor syntax highlighting en debugging, maar het is ook een van de weinige editors die code completion aanbied voor python. Het gebrek aan goede editors kan verklaard worden doordat python, zoals in \ref{software_python} een dynamische taal is, dit maakt het bieden van code completion een stuk moeilijker.

\subsection{Visualisation Toolkit (VTK)}
Voor het implementeren van de benodigde Image Processing Technieken maken we gebruik van de Visualisation Toolkit (VTK)\cite{vtk}. 
VTK is een open source toolkit voor 3D graphics. 
In ons geval zetten we de toolkit in voor 2D graphics. 
VTK bestaat uit een C++ class bibliotheek en onder andere een Python interface laag.\cite{vtk2} 
Er zitten uitgebreide mogelijkheden in op het gebied van Image Processing en visualisatie.
In de VTK zitten de benodigde filters om Generalized Procrustes Analysis en Principal Component Analysis uit te voeren. 
Verder bevat de toolkit een implementatie van de Thin Plate Spline transformatie. 
Door de z-co\"{o}rdinaten op 0 te zetten kan VTK in de 2D ruimte worden gebruikt. 

\subsection{Python Imaging Library (PIL)}
Om de wat simpelere Image Processing technieken te implementeren maken we gebruik van de Python Imaging Library (PIL)\cite{pil}. 
Er is een handbook beschikbaar waarin de functionaliteiten besproken worden.\cite{pilhandbook} 
Dit kan onder andere ingezet worden voor het resizen van images, puntoperaties, filtering, colour space conversies, rotatie en transformaties. 

\subsection{JavaScript}
Om een mooie en gebruiksvriendelijke user-interface te krijgen, gaan wij gebruik maken van JavaScript.
Hiervoor zijn er een aantal bibliotheken beschikbaar die ons gaan helpen (Scriptaculous\cite{scriptaculous} en Prototype\cite{prototype}).
Prototype gaan we onder andere gebruiken vanwege de mooie mogelijkheden om met AJaX om te gaan, en de handige manieren om objecten aan te maken.
De Scriptaculous bibliotheek biedt ons verschillende mogelijkheden voor visuele toepassingen binnen een website, zoals sliders en draggables

\subsection{Flash}
%TODO: Bastiaan, your expertise, show us it.
