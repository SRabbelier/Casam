\section{Software}
\label{Software}
In dit hoofdstuk bespreken we de softwaretechnologi/"{e}n die we gaan gebruiken om de applicatie te maken.
We beginnen met de basis van de applicatie door eerst de IDE, Python en Django te bespreken.
Hierna bespreken we de 2 gebruikte bibliotheken, VTK en PIL.
Tot slot bespreken we kort JavaScript en Flash.

\subsection{Python}
Het programma is geschreven in python, een dynamische programeertaal met goede ondersteuning voor externe bibliotheken in python zelf, maar ook in C of C++. De taal is zo opgebouwd dat het makkelijk is om prototypes te maken en die later door middel van refactoring om te schrijven naar een compleet eindproduct. Door de ondersteuning van onder andere objectgeori\"{e}nteerd programmeren maar ook functioneel programmeren is het voor vele programmeurs een makkelijke taal om te leren en een krachtige taal om te gebruiken.

\subsection{Django}
Met behulp van Django wordt Python een taal waarmee eenvoudig web applicaties geschreven kunnen worden zonder dat de Python code zelf vervuild wordt met html en css. De applicatie maakt gebruik van HTML-templates waarin met behulp van een specifieke syntax data van de Python kant ingevoegd kan worden. Tevens kan door middel van overerving een consistente webapplicatie geschreven worden zonder dat er (HTML) code gekopieerd en geplakt hoeft te worden. Django heeft ook een uitgebreide forms-module waarmee HTML-formulieren gemaakt en verwerkt kunnen worden. Onderdeel van dit verwerken is ook het valideren van de gegevens die door de gebruiker aangeleverd worden. Tenslotte heeft Django ondersteuning voor het escapen van html zodat door de gebruiker ingevoerde tekst nooit kan resulteren in bijvoorbeeld cross-site scripting (XSS).

\subsection{Integrated Development Environment (IDE): Eclipse}
Eclipse, in samenwerking met de pydev plugin, is op het moment een van de beste Python editors. Het heeft niet alleen ondersteuning voor syntax highlighting en debugging, maar het is ook een van de weinige editors die code-completion aanbied voor Python. Het gebrek aan goede editors kan verklaard worden doordat Python een dynamische taal is, dit maakt het bieden van code completion een stuk moeilijker.

\subsection{Visualisation ToolKit (VTK)}
Voor het implementeren van de benodigde Image Processing Technieken maken we gebruik van de Visualisation ToolKit (VTK)\cite{vtk}.
VTK is een open source toolkit voor 3D graphics.
In ons geval zetten we de toolkit in voor 2D graphics.
VTK bestaat uit een C++ bibliotheek en verschillende interface-lagen waaronder een voor Python.\cite{vtk2}
Er zitten uitgebreide mogelijkheden in op het gebied van Image Processing en visualisatie, zoals de benodigde filters om Generalized Procrustes Analysis en Principal Component Analysis uit te voeren.
Verder bevat de toolkit een implementatie van de Thin Plate Spline transformatie.
Door de z-co\"{o}rdinaten op 0 te zetten kan VTK in de 2D ruimte worden gebruikt.

\subsection{Python Imaging Library (PIL)}
Om de wat simpelere Image Processing technieken te implementeren maken we gebruik van de Python Imaging Library (PIL)\cite{pil}.
Er is een handboek beschikbaar waarin de functionaliteiten besproken worden.\cite{pilhandbook}
Dit kan onder andere ingezet worden voor het resizen van images, puntoperaties, filtering, colour space conversies, rotatie en transformaties. Waar VTK zich specialiseert in meer high-level operaties is PIL juist meer geschikt om de wat eenvoudigere operaties uit te voeren.

\subsection{JavaScript}
Om een mooie en gebruiksvriendelijke user-interface te krijgen, gaan wij gebruik maken van JavaScript.
Hiervoor gebruiken we een aantal bibliotheken (Scriptaculous\cite{scriptaculous} en Prototype\cite{prototype}).
Prototype gaan we gebruiken vanwege de mogelijkheden om met AJaX om te gaan, en de manieren om objecten aan te maken.
De Scriptaculous bibliotheek biedt ons verschillende mogelijkheden voor visuele toepassingen binnen de applicatie, zoals sliders en draggable objecten.

\subsection{Flash}
Een deel van het systeem bestaat uit het kunnen aangeven van belangrijke gebieden in foto's door in de browser te tekenen. Een Flash-applicatie kan gebruikt worden om dit te implementeren. Een Flash-applicatie heeft een uitgebreid layeringsysteem, waarmee afbeeldingen en lege lagen geladen kunnen worden. In deze lagen kan live getekend worden. Omdat er op goede wijze events worden afgevangen, is het mogelijk om met behulp van de muis lijnen te laten tekenen door de gebruiker.

Flash ondersteunt ook uitgebreide communicatie: het is mogelijk om data te versturen naar de server. Dit alles maakt Flash zeer geschikt voor onze applicatie. We hebben gebruik gemaakt van Flash 8 van Macromedia, met als scriptingtaal ActionScript 2.0. Deze keuze is gemaakt omdat deze versie voor handen was en voldoet aan de eisen.
